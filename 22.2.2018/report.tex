%
% This is the LaTeX template file for lecture notes for CS267,
% Applications of Parallel Computing.  When preparing 
% LaTeX notes for this class, please use this template.
%
% To familiarize yourself with this template, the body contains
% some examples of its use.  Look them over.  Then you can
% run LaTeX on this file.  After you have LaTeXed this file then
% you can look over the result either by printing it out with
% dvips or using xdvi.
%

\documentclass[twoside]{article}
\setlength{\oddsidemargin}{0.25 in}
\setlength{\evensidemargin}{-0.25 in}
\setlength{\topmargin}{-0.6 in}
\setlength{\textwidth}{6.5 in}
\setlength{\textheight}{8.5 in}
\setlength{\headsep}{0.75 in}
\setlength{\parindent}{0 in}
\setlength{\parskip}{0.1 in}

%
% ADD PACKAGES here:
%

\usepackage{amsfonts,graphicx}
%\usepackage{amsmath}
\usepackage{algorithm}
%\usepackage[noend]{algpseudocode}
\usepackage{verbatim}

\usepackage[T1]{fontenc}
\usepackage[utf8]{inputenc}
\usepackage[english]{babel}
%\usepackage[margin=1in]{geometry}
\usepackage{natbib}
%\setcitestyle{nocompress}
%\bibpunct{(}{)}{;}{a}{}{,}

% \usepackage[nocompress]{cite}
%\usepackage{graphicx}
%\usepackage[cmex10]{amsmath}
%\usepackage{authblk}
% \usepackage[colorinlistoftodos,prependcaption,textsize=tiny]{todonotes}
\usepackage{macros}
\usepackage{times}


%
% The following commands set up the lecnum (lecture number)
% counter and make various numbering schemes work relative
% to the lecture number.
%
\newcounter{lecnum}
\renewcommand{\thepage}{\arabic{page}}
\renewcommand{\thesection}{\arabic{section}}
\renewcommand{\theequation}{\arabic{equation}}
\renewcommand{\thefigure}{\arabic{figure}}
\renewcommand{\thetable}{\arabic{table}}

%
% The following macro is used to generate the header.
%
\newcommand{\lecture}[4]{
   \pagestyle{myheadings}
   \thispagestyle{plain}
   \newpage
   \setcounter{lecnum}{#1}
   \setcounter{page}{1}
   \noindent
   \begin{center}
   \framebox{
      \vbox{\vspace{2mm}
    %\hbox to 6.28in { {\bf Report on my work with Adobe
		%\hfill Spring 2018} }
       \vspace{4mm}
       \hbox to 6.28in { {\Large \hfill Generalized Latent Bandits \hfill} }
       \vspace{2mm}
       \hbox to 6.28in { {\it Adobe Advisor(s): Branislav Kveton, Anup Rao \hfill Intern: Subhojyoti Mukherjee} }
      \vspace{2mm}}
   }
   \end{center}
   %\markboth{Lecture #1: #2}{Lecture #1: #2}

   {\bf Disclaimer}: {\it These notes have not been subjected to the
   usual scrutiny reserved for formal publications.  They may be distributed
   outside this communication only with the permission of the Advisor(s).}
   \vspace*{4mm}
}
%
% Convention for citations is authors' initials followed by the year.
% For example, to cite a paper by Leighton and Maggs you would type
% \cite{LM89}, and to cite a paper by Strassen you would type \cite{S69}.
% (To avoid bibliography problems, for now we redefine the \cite command.)
% Also commands that create a suitable format for the reference list.
\renewcommand{\cite}[1]{[#1]}
\def\beginrefs{\begin{list}%
        {[\arabic{equation}]}{\usecounter{equation}
         \setlength{\leftmargin}{2.0truecm}\setlength{\labelsep}{0.4truecm}%
         \setlength{\labelwidth}{1.6truecm}}}
\def\endrefs{\end{list}}
\def\bibentry#1{\item[\hbox{[#1]}]}

%Use this command for a figure; it puts a figure in wherever you want it.
%usage: \fig{NUMBER}{SPACE-IN-INCHES}{CAPTION}
\newcommand{\fig}[3]{
			\vspace{#2}
			\begin{center}
			Figure #1:~#3
			\end{center}
	}
% Use these for theorems, lemmas, proofs, etc.
%\newtheorem{theorem}{Theorem}
%\newtheorem{lemma}[theorem]{Lemma}
%\newtheorem{proposition}[theorem]{Proposition}
%\newtheorem{claim}[theorem]{Claim}
%\newtheorem{corollary}[theorem]{Corollary}
%\newtheorem{definition}[theorem]{Definition}
%\newenvironment{proof}{{\bf Proof:}}{\hfill\rule{2mm}{2mm}}

% **** IF YOU WANT TO DEFINE ADDITIONAL MACROS FOR YOURSELF, PUT THEM HERE:

%\newcommand\E{\mathbb{E}}


\begin{document}
%FILL IN THE RIGHT INFO.
%\lecture{**LECTURE-NUMBER**}{**DATE**}{**LECTURER**}{**SCRIBE**}
\lecture{1}{A - Title}{Lecturer Name}{scribe-name}
%\footnotetext{These notes are partially based on those of Nigel Mansell.}

% **** YOUR NOTES GO HERE:

% Some general latex examples and examples making use of the
% macros follow.  
%**** IN GENERAL, BE BRIEF. LONG SCRIBE NOTES, NO MATTER HOW WELL WRITTEN,
%**** ARE NEVER READ BY ANYBODY.
%This lecture's notes illustrate some uses of
%various \LaTeX\ macros.  
%Take a look at this and imitate.

\begin{abstract}
To be written.
\end{abstract}


\section{Introduction}
In this paper, we study the problem of recommending the best items to users who are coming sequentially. The learner has access to very less prior information about the users and it has to adapt quickly to the user preferences and suggest the best item to each user. Furthermore, we consider the setting where users are grouped into clusters and within each cluster the users have the same choice of the best item, even though their quality of preference may be different for the best item. These clusters along with the choice of the best item for each user are unknown to the learner.  Also, we assume that each user has a single best item preference.

	This complex problem can be conceptualized as a low rank stochastic bandit problem where there are $K$ users, $L$ items and the users are coming sequentially. The reward matrix, denoted by $\bar{R}\in [0,1]^{K\times L}$,  generating the rewards for user, item pair has a low rank structure. The online learning game proceeds as follows, at every timestep $t$,  nature reveals one user (or row) from $\bar{R}$ where user is denoted by $i_t$. The learner selects one item (or column) from $\bar{R}$, where the item is denoted by $j_t$. Then the learner receives one noisy feedback $X_{i_t,j_t}\sim\mathcal{G}(\bar{R}_{i_t,j_t},\sigma^2)$ from this reward matrix, where $\mathcal{G}$ is a distribution over the entries in $\bar{R}$ and $\E[X_{i_t,j_t}] = \bar{R}_{i_t,j_t}$. Then the goal of the learner is to minimize the cumulative regret by quickly identifying the best item $j^*_t$ for each $i_t\in \bar{R}$ where $\bar{R}_{i_t,j^*_t} = \argmax _{j\in[L]}\lbrace \bar{R}_{i_t,j_t} \rbrace$. Also, there exist $C$ clusters, indexed from $\lbrace 1,2,\ldots, C\rbrace \in\C$ such that for each $i_t\in U$, 


\subsection{Notations, Problem Formulation and Assumptions}
%Extending the discussion in the previous section, we propose a Generalized Latent Bandit model where each $v_{a,b}$ can be considered as a mixture of several $\lbrace v_{a,c}\rbrace_{c\in\C}$. This is a harder problem than Latent Bandit model because now the rows in each cluster are not exactly identical but the index of the optimal arm (column) is similar in each row.
	
	We define $[n] = \lbrace 1,2,\ldots, n\rbrace$ and for any two sets $A$ and $B$, $A^B$ denotes the set of all vectors who take values from $A$ and are indexed by $B$. Let, $R\in [0,1]^{K\times L}$ denote any matrix, then $R(I,:)$ denote any submatrix of $k$ rows such that $I\in[K]^k$ and similarly $R(:,J)$ denote any submatrix of $j$ columns such that $J\in[L]^{j}$.
	
	Let $\bar{R}$ be reward matrix of  dimension $K\times L$ where $K$ is the number of user or rows and $L$ is the number of arms or columns. Also, let us assume that this matrix  $\bar{R}$ has a low rank structure of rank $d << \min\lbrace L,K\rbrace$. Let $U$ and $V$ denote the latent matrices for the users and items, which are not visible to the learner such that,
\begin{align*}
	\bar{R} = UV^{\intercal} \textbf{ \hspace*{4mm}   s.t.   \hspace*{4mm}} U\in [ \mathbb{R}^+ ]^{K\times d} \textbf{, } V\in  [0,1]^{L\times d} 
\end{align*}	  
	
	Furthermore, we put a constraint on $V$ such that, $\forall j\in [L]$, $ \norm{V(j,:)}_1 \leq 1$. 
	
	
\begin{assumption}
\label{assm:1}
We assume that there exists $d$-column base factors, denoted by $V(J^*,:)$, such that all rows of $V$ can be written as a convex combination of $V(J^*,:)$ and the zero vector and $J^* = [d]$. We denote the column factors by $V^* = V(J*,:)$. Therefore, for any $i\in [L]$, it can be represented by
\begin{align*}
V(i,:) = a_i V(J^*,:) , 
\end{align*}
where $\exists a_i\in [0,1]^{d}$ and $ \norm{a_i}_1 \leq 1$.
\end{assumption}

%In this paper, in addition to the noisy setting explained in section \ref{intro} we first analyze the proposed algorithm in the easier noise free setting. In the noise free setting, the nature reveals the row $i_t$, and when the learner selects the column $j_t$, it observes the mean of the distribution $\bar{R}(i_t,j_t)$.

%\begin{assumption}
%\label{assm:round-robin}
%We assume that nature is revealing the user $i$ in $\bar{R}(i,:), \forall i\in [K]$  in a Round-Robin fashion such that at timestep $t$, nature reveals $i_t = (t \mod K) + 1$.
%\end{assumption}

\begin{assumption}
\label{assm:d-items}
For each user $i_t$ revealed by the nature at round $t$, the learner is allowed to select atmost $d$-items, where $d$ is the rank of the matrix $\bar{R}$.
\end{assumption}

The above assumption \ref{assm:d-items} can be conceptualized in this real-world scenario where the learner has  to suggest movies to users and each movie belongs to a different genre (say thriller, romance, comedy, etc). So, the learner can suggest $d$ movies belonging to different genres to each user, and the user can click one, or both, or none of the recommended movies.


The main goal of the learning agent is to minimize the cumulative regret until the end of horizon $n$. We define the cumulative regret, denoted by $\mathcal{R}_n$ as,

\begin{align*}
\mathcal{R}_n = \sum_{t=1}^{n}\bigg\lbrace \sum_{z=1}^{d} \bigg( r_{z,t}\left(i_{t}, j^*_{z,t} \right) - r_{z, t}\left( i_{t}, j_{z,t}\right)\bigg)\bigg\rbrace
\end{align*}

where, $j^*_{z,t} = \argmax_{j\in [L]}\lbrace \bar{R}(i_t,j)\rbrace$ and $j_{z,t}$ be the suggestion of the learner for the $i_t$ -th user for  $z=1,2,\ldots, d$. Note that $r_{t}\left(i_t, j^*_{z,t} \right)\sim \mathcal{N}(\bar{R}\left(i_t, j^*_{z,t} \right),\sigma^2)$ and $r_{t}\left(i_t, j_{z,t} \right)\sim \mathcal{N}(\bar{R}\left(i_t, j_{z,t} \right), \sigma^2)$. Taking expectation over both sides, we can show that,

\begin{align*}
\E[\mathcal{R}_n] = \E\left [ \sum_{t=1}^{n}\bigg\lbrace\sum_{z=1}^{d} \bigg( r_{z,t}\left(i_t, j^*_{z,t} \right) - r_{z, t}\left( i_t, j_{z, t}\right)\bigg)\bigg\rbrace\right] = \E\left [ \sum_{t=1}^{T} \sum_{z=1}^{d} \bigg( N_{i_t,j_{z,t}}\bigg) \right ]\Delta_{i_t,j_{z,t}}
\end{align*}

where, $\Delta_{i_t,j_{z,t}} = \bar{R}(i_t,j^*_{z,t}) - \bar{R}(i_t,j_{z,t})$ and $N_{i_t,j_{z,t}}$ is the number of times the learner has observed the $j_{z,t}$-th item for the $i_t$-th user for $z=1,2,\ldots, d$. Let, $\Delta = \min_{i\in[K],j\in[L]}\lbrace \Delta_{i,j}\rbrace$ be the minimum gap over all the user, item pair in $\bar{R}$.


%\subsection{Notations and Assumptions}
%\begin{assumption}
\label{assm:1}
We assume that there exists $d$-column base factors, denoted by $V(J^*,:)$, such that all row of $V$ can be written as a convex combination of $V(J^*,:)$ and the zero vector and $J^* = [d]$. We denote the column factors by $V^* = V(J*,:)$. Therefore, for any $i\in [L]$, it can be represented by
\begin{align*}
V(i,:) = a_i V(J^*,:) , 
\end{align*}
where $\exists a_i\in [0,1]^{d}$ and $ ||a_i||_1 \leq 1$.
\end{assumption}
	
\subsection{Related Works}
In \citet{maillard2014latent} the authors propose the Latent Bandit model where there are two sets: 1) set of arms denoted by $\A$ and 2) set of types denoted by $\B$ which contains the latent information regarding the arms. The latent information for the arms are modeled such that the set $\B$ is assumed to be partitioned into $|C|$ clusters, indexed by $\B_1, \B_2, \ldots, \B_C \in \C$ such that the distribution $v_{a,b}, a\in\A, b\in\B_c$ across each cluster is same.  Note, that the identity of the cluster is unknown to the learner. At every timestep $t$, nature selects a type $b_t\in\B_c$ and then the learner selects an arm $a_t\in\A$ and observes a reward $X_{a,t}$ from the distribution $v_{a,b}$.
	
	Another way to look at this problem is to imagine a matrix of dimension $|A|\times |B|$ where again the rows in $\B$ can be partitioned into $|C|$ clusters, such that the distribution across each of this clusters are same. Now, at every timestep $t$ one of this row is revealed to the learner and it chooses one column such that the $v_{a,b}$ is one of the $\lbrace v_{a,c}\rbrace_{c\in\C}$ and the reward for that arm and the user is revealed to the learner.
	
	This is actually a much simpler approach because note that the distributions across each of the clusters $\lbrace v_{a,c}\rbrace_{c\in\C}$ are identical and estimating one cluster distribution will reveal all the information of the users in each cluster.
	

\section{Contributions}
To be written.


\section{Proposed Algorithms}
\subsection{Noise-Free Setting}

In the noise-free setting, in addition to Assumption \ref{assm:1} and Assumption \ref{assm:1} we assume two further assumptions.

\begin{assumption}
\label{assm:round-robin}
We assume that nature is revealing the $i$ of $\bar{R}(i,:), \forall i\in [K]$  in a Round-Robin fashion.
\end{assumption}

\begin{assumption}
\label{assm:d-indep}
We assume that any $d$ sets of rows or columns are independent in $\bar{R}$.
\end{assumption}

\algblock{ArmElim}{EndArmElim}
\algnewcommand\algorithmicArmElim{\textbf{\em Arm Elimination}}
 \algnewcommand\algorithmicendArmElim{}
\algrenewtext{ArmElim}[1]{\algorithmicArmElim\ #1}
\algrenewtext{EndArmElim}{\algorithmicendArmElim}

\algblock{ResParam}{EndResParam}
\algnewcommand\algorithmicResParam{\textbf{\em Reset Parameters}}
 \algnewcommand\algorithmicendResParam{}
\algrenewtext{ResParam}[1]{\algorithmicResParam\ #1}
\algrenewtext{EndResParam}{\algorithmicendResParam}

\algblock{ColRec}{EndColRec}
\algnewcommand\algorithmicColRec{\textbf{\em Column Reconstruction}}
 \algnewcommand\algorithmicendColRec{}
\algrenewtext{ColRec}[1]{\algorithmicColRec\ #1}
\algrenewtext{EndColRec}{\algorithmicendColRec}

\algblock{ColElim}{EndColElim}
\algnewcommand\algorithmicColElim{\textbf{\em Column Elimination}}
 \algnewcommand\algorithmicendColElim{}
\algrenewtext{ColElim}[1]{\algorithmicColElim\ #1}
\algrenewtext{EndColElim}{\algorithmicendColElim}

\algblock{Explore}{EndExplore}
\algnewcommand\algorithmicExplore{\textbf{\em Explore}}
 \algnewcommand\algorithmicendExplore{}
\algrenewtext{Explore}[1]{\algorithmicExplore\ #1}
\algrenewtext{EndExplore}{\algorithmicendExplore}


\algblock{Exploit}{EndExploit}
\algnewcommand\algorithmicExploit{\textbf{\em Exploit}}
 \algnewcommand\algorithmicendExploit{}
\algrenewtext{Exploit}[1]{\algorithmicExploit\ #1}
\algrenewtext{EndExploit}{\algorithmicendExploit}




\begin{algorithm}[!th]
\caption{Noise-Free GLB}
\label{alg:NFGLB}
\begin{algorithmic}
\State {\bf Input:} Time horizon $T$, $Rank(\bar{R}) = d$.
%\State {\bf Definition:} Select .
\State {\bf Initialization:} Randomly select $d$ columns with uniform probability and $J\leftarrow \lbrace d \rbrace$ , $\B \leftarrow \emptyset$ and $\forall I\in [K], J_0\in [L], \hat{R}(I,J_0) \leftarrow 0$.
%$\forall j\in J$, and 
\State Then, for all the rows $I\in[K]$ observe $\bar{R}(I,J)$ and update  $\hat{R}(I,J) \leftarrow \bar{R}(I,J)$.
\State \textbf{Explore: } Randomly select test column $c \in \A \setminus (\B + J)$ with uniform probability and set $r_c \leftarrow 0$.
%\State $r_c$ := $1$
\For{$t=Kd+1,..,T$}	
\State Nature reveals $i_t$ such that $i_t \leftarrow (t \mod K) + 1$ (Round-Robin).
\If{$|\B| < L - d$} \textbf{Explore}
\If{$r_c < d$}
\State Choose $c$, observe $\bar{R}(i_t,c)$ and $\hat{R}(i_t,c)\leftarrow \bar{R}(i_t,c)$.
\State $r_c \leftarrow r_c + 1$
\Else
\ColElim
\State \If{$\forall i\in [K]: \max_{j\in J} \hat{R}(i,j)  \geq  \hat{R}(i,c)$} 
\State  $\B \leftarrow \B \cup c$ (Eliminate $c$ )
\State Randomly select another test column $c \in \A \setminus (\B + J)$ with uniform probability.
\State $r_c \leftarrow 0$
\Else
\State $\exists j'\in J: \forall i\in [K]: \max_{j\in J \setminus{ j'} \cup c} \hat{R}(i,j) \geq \hat{R}(i,j')$
\State $\B \leftarrow \B \cup j' $ (Eliminate $j'$)
\State $J\leftarrow J \setminus{ j'} \cup c$ (Add $c$ as best candidate column)
\State $r_c \leftarrow d - K $ (Fully explore new best candidate column)
\EndIf
\EndColElim
\EndIf
\ElsIf{$|\B| = L - d$} \textbf{Exploit}
\State  Select column $j_t^*$ and observe $\bar{R}(i_t,j_t^*)$ where $j_t^* = \argmax_{j\in[J]} {\hat{R}(i_t,j)}$.
\EndIf
\EndFor
\end{algorithmic}
%\vspace*{-0.42em}
\end{algorithm}





\section{Main Results}
%\begin{lemma}
%For any arbitrary row $i\in[K]$, 
%\begin{align*}
%\argmax_{j\in[L]} U(i,:){V(j,:)}^{\intercal} \leq \argmax_{j\in[d]} U(i,:){V(j,:)}^{\intercal} . 
%\end{align*}
%\end{lemma}

\subsection{Regret Bound of GLBUCB}




\section{Proofs}
\begin{proof}
Considering any arbitrary row $i\in [K]$, we can show that,
\begin{align*}
\argmax_{j\in[L]} U(i,:){V(j,:)}^{\intercal}  &= U(i,:)V(j^*(i),:)^{\intercal}\\
& \overset{(a)}{=} U(i,:)\left(a_{j^*(i)}V(J^*,:)\right)^{\intercal}\\
& = \sum_{k=1}^{d}a_{j^*(i)}(k)U(i,:)V(j^*(i),:)^{\intercal}\\
& \leq \argmax a_{j^*(i)}(k)U(i,:)V(k,:)^{\intercal} \\
& \leq \argmax_{k\in[d]}U(i,:){V(k,:)}^{\intercal}   ,
\end{align*}
where $(a)$ is from Assumption \ref{assm:1}.
\end{proof}




\section{Experiments}




%\definecolor{blueaccent}{RGB}{0,150,214}
%\definecolor{greenaccent}{RGB}{0,139,43}
%\definecolor{purpleaccent}{RGB}{130,41,128}
%\definecolor{orangeaccent}{RGB}{240,83,50}
%
%\addplot+[greenaccent,mark=*,mark options={fill=greenaccent}] table{results/NewExpt1/Expt3/comp_subsampled_CustomMOSS0RR1.txt};
%		\addplot+[purpleaccent,mark=*,mark options={fill=purpleaccent}] table{results/NewExpt1/Expt3/comp_subsampled_CustomOCUCB0RR1.txt};
%		\addplot+[orangeaccent,mark=*,mark options={fill=orangeaccent}] table{results/NewExpt1/Expt3/comp_subsampled_CustomEXP30RR1.txt};

\begin{figure}[!th]
\centering
\begin{tabular}{c}
\setlength{\tabcolsep}{0.1pt}
\subfigure[0.25\textwidth][Expt-$1$: $2048$ Users, $64$ Bernoulli-distributed arms, Round-Robin, Noisy Setting, Rank $2$, equal sized clusters]
    %with $r_{i_{{i}\neq {*}}}=0.07$ and $r^{*}=0.1$
    {
    		\pgfplotsset{
		tick label style={font=\Large},
		label style={font=\Large},
		legend style={font=\Large},
		ylabel style={yshift=5pt},
		%legend style={legendshift=32pt},
		}
        \begin{tikzpicture}[scale=0.8]
      	\begin{axis}[
		xlabel={timestep},
		ylabel={Cumulative Regret},
		grid=major,
        %clip mode=individual,grid,grid style={gray!30},
        clip=true,
        %clip mode=individual,grid,grid style={gray!30},
  		legend style={at={(0.5,1.4)},anchor=north, legend columns=3} ]
      	% UCB
		
		
		\addplot table{results/NewExpt1/Expt0/comp_subsampled_MetaEXP0RR2.txt};
		\addplot table{results/NewExpt1/Expt0/comp_subsampled_MetaEXP0RR1.txt};
      	\legend{LRG, LRUCB} 
      	\end{axis}
      	\end{tikzpicture}
  		\label{fig:0}
    }
 \end{tabular}
    \caption{A comparison of the cumulative regret by MRLG and MRLUCB. }
    \label{fig:karmed}
    \vspace*{-1em}
\end{figure}

\begin{figure}[!th]
\centering
\begin{tabular}{cc}
\setlength{\tabcolsep}{0.1pt}
\subfigure[0.25\textwidth][Expt-$1$: $2048$ Users, $64$ Bernoulli-distributed arms, Round-Robin, Noisy Setting, Rank $2$, equal sized clusters]
    %with $r_{i_{{i}\neq {*}}}=0.07$ and $r^{*}=0.1$
    {
    		\pgfplotsset{
		tick label style={font=\Large},
		label style={font=\Large},
		legend style={font=\Large},
		ylabel style={yshift=5pt},
		%legend style={legendshift=32pt},
		}
        \begin{tikzpicture}[scale=0.8]
      	\begin{axis}[
		xlabel={timestep},
		ylabel={Cumulative Regret},
		grid=major,
        %clip mode=individual,grid,grid style={gray!30},
        clip=true,
        %clip mode=individual,grid,grid style={gray!30},
        cycle list name=exotic,
  		legend style={at={(0.5,1.4)},anchor=north, legend columns=3} ]
      	% UCB
		\addplot table{results/NewExpt1/Expt2/comp_subsampled_CustomUCB0RR1.txt};
		\addplot table{results/NewExpt1/Expt2/comp_subsampled_CustomOCUCB0RR1.txt};
		\addplot table{results/NewExpt1/Expt2/comp_subsampled_MetaEXP0RR2.txt};
		\addplot table{results/NewExpt1/Expt2/comp_subsampled_CustomTS0RR1.txt};
		\addplot table{results/NewExpt1/Expt2/comp_subsampled_CustomMOSS0RR1.txt};
		\addplot table{results/NewExpt1/Expt2/comp_subsampled_MetaEXP0RR1.txt};
		\addplot table{results/NewExpt1/Expt2/comp_subsampled_CustomEXP30RR1.txt};
		\addplot table{results/NewExpt1/Expt2/comp_subsampled_CustomKLUCB0RR1.txt};
      	\legend{UCB1,OCUCB,LRG,TS,MOSS,LRUCB, EXP3,KLUCB} 
      	\end{axis}
      	\end{tikzpicture}
  		\label{fig:1}
    }
    &
    \subfigure[0.25\textwidth][Expt-$2$: $2048$ Users, $64$ Bernoulli-distributed arms, Round-Robin, Noisy Setting, Rank $2$, un-equal sized clusters, 70:30 split]
    %with $r_{i_{{i}\neq {*}}}=0.07$ and $r^{*}=0.1$
    {
    		\pgfplotsset{
		tick label style={font=\Large},
		label style={font=\Large},
		legend style={font=\Large},
		ylabel style={yshift=5pt},
		%legend style={legendshift=32pt},
		}
        \begin{tikzpicture}[scale=0.8]
      	\begin{axis}[
		xlabel={timestep},
		ylabel={Cumulative Regret},
		grid=major,
        %clip mode=individual,grid,grid style={gray!30},
        clip=true,
        %clip mode=individual,grid,grid style={gray!30},
        cycle list name=exotic,
  		legend style={at={(0.5,1.4)},anchor=north, legend columns=3} ]
      	% UCB
		\addplot table{results/NewExpt1/Expt3/comp_subsampled_CustomUCB0RR1.txt};
		\addplot table{results/NewExpt1/Expt3/comp_subsampled_CustomOCUCB0RR1.txt};
		\addplot table{results/NewExpt1/Expt3/comp_subsampled_MetaEXP0RR2.txt};
		\addplot table{results/NewExpt1/Expt3/comp_subsampled_CustomTS0RR1.txt};
		\addplot table{results/NewExpt1/Expt3/comp_subsampled_CustomMOSS0RR1.txt};
		\addplot table{results/NewExpt1/Expt3/comp_subsampled_MetaEXP0RR1.txt};
		\addplot table{results/NewExpt1/Expt3/comp_subsampled_CustomEXP30RR1.txt};
		\addplot table{results/NewExpt1/Expt3/comp_subsampled_CustomKLUCB0RR1.txt};
      	\legend{UCB1,OCUCB,LRG,TS,MOSS,LRUCB, EXP3,KLUCB} 
      	\end{axis}
      	\end{tikzpicture}
  		\label{fig:2}
    }
    \\
    \subfigure[0.25\textwidth][Expt-$3$: $4096$ Users, $128$ Bernoulli-distributed arms, Round-Robin, Noisy Setting, Rank $2$, equal sized clusters]
    %with $r_{i_{{i}\neq {*}}}=0.07$ and $r^{*}=0.1$
    {
    		\pgfplotsset{
		tick label style={font=\Large},
		label style={font=\Large},
		legend style={font=\Large},
		ylabel style={yshift=5pt},
		%legend style={legendshift=32pt},
		}
        \begin{tikzpicture}[scale=0.8]
      	\begin{axis}[
		xlabel={timestep},
		ylabel={Cumulative Regret},
		grid=major,
        %clip mode=individual,grid,grid style={gray!30},
        clip=true,
        %clip mode=individual,grid,grid style={gray!30},
        cycle list name=exotic,
  		legend style={at={(0.5,1.4)},anchor=north, legend columns=3} ]
      	% UCB
		\addplot table{results/NewExpt1/Expt4/comp_subsampled_CustomUCB0RR1.txt};
		\addplot table{results/NewExpt1/Expt4/comp_subsampled_CustomOCUCB0RR1.txt};
		\addplot table{results/NewExpt1/Expt4/comp_subsampled_MetaEXP0RR2.txt};
		\addplot table{results/NewExpt1/Expt4/comp_subsampled_CustomTS0RR1.txt};
		\addplot table{results/NewExpt1/Expt4/comp_subsampled_CustomMOSS0RR1.txt};
		\addplot table{results/NewExpt1/Expt4/comp_subsampled_MetaEXP0RR1.txt};
		\addplot table{results/NewExpt1/Expt4/comp_subsampled_CustomEXP30RR1.txt};
      	\addplot table{results/NewExpt1/Expt4/comp_subsampled_CustomKLUCB0RR1.txt};
      	\legend{UCB1,OCUCB,LRG,TS,MOSS,LRUCB, EXP3,KLUCB} 
      	\end{axis}
      	\end{tikzpicture}
  		\label{fig:3}
    }
    &
    \subfigure[0.25\textwidth][Expt-$4$: $4096$ Users, $128$ Bernoulli-distributed arms, Round-Robin, Noisy Setting, Rank $2$, un-equal sized clusters, 80:20 split]
    %with $r_{i_{{i}\neq {*}}}=0.07$ and $r^{*}=0.1$
    {
    		\pgfplotsset{
		tick label style={font=\Large},
		label style={font=\Large},
		legend style={font=\Large},
		ylabel style={yshift=5pt},
		%legend style={legendshift=32pt},
		}
        \begin{tikzpicture}[scale=0.8]
      	\begin{axis}[
		xlabel={timestep},
		ylabel={Cumulative Regret},
		grid=major,
        %clip mode=individual,grid,grid style={gray!30},
        clip=true,
        %clip mode=individual,grid,grid style={gray!30},
        cycle list name=exotic,
  		legend style={at={(0.5,1.4)},anchor=north, legend columns=3} ]
      	% UCB
		\addplot table{results/NewExpt1/Expt5/comp_subsampled_CustomUCB0RR1.txt};
		\addplot table{results/NewExpt1/Expt5/comp_subsampled_CustomOCUCB0RR1.txt};
		\addplot table{results/NewExpt1/Expt5/comp_subsampled_MetaEXP0RR2.txt};
		\addplot table{results/NewExpt1/Expt5/comp_subsampled_CustomTS0RR1.txt};
		\addplot table{results/NewExpt1/Expt5/comp_subsampled_CustomMOSS0RR1.txt};
		\addplot table{results/NewExpt1/Expt5/comp_subsampled_MetaEXP0RR1.txt};
		\addplot table{results/NewExpt1/Expt5/comp_subsampled_CustomEXP30RR1.txt};
      	\addplot table{results/NewExpt1/Expt5/comp_subsampled_CustomKLUCB0RR1.txt};
      	\legend{UCB1,OCUCB,LRG,TS,MOSS,LRUCB, EXP3,KLUCB} 
      	\end{axis}
      	\end{tikzpicture}
  		\label{fig:4}
    }
    \end{tabular}
    \caption{A comparison of the cumulative regret incurred by the various bandit algorithms. }
    \label{fig:karmed1}
    \vspace*{-1em}
\end{figure}



\section{Conclusions and Future Direction}
To be written.


%\clearpage
%\newpage
%\bibliographystyle{plainnat}
\bibliographystyle{apalike}
\bibliography{biblio}

\end{document}





