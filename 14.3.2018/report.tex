%
% This is the LaTeX template file for lecture notes for CS267,
% Applications of Parallel Computing.  When preparing 
% LaTeX notes for this class, please use this template.
%
% To familiarize yourself with this template, the body contains
% some examples of its use.  Look them over.  Then you can
% run LaTeX on this file.  After you have LaTeXed this file then
% you can look over the result either by printing it out with
% dvips or using xdvi.
%

\documentclass[twoside]{article}
\setlength{\oddsidemargin}{0.25 in}
\setlength{\evensidemargin}{-0.25 in}
\setlength{\topmargin}{-0.6 in}
\setlength{\textwidth}{6.5 in}
\setlength{\textheight}{8.5 in}
\setlength{\headsep}{0.75 in}
\setlength{\parindent}{0 in}
\setlength{\parskip}{0.1 in}

%
% ADD PACKAGES here:
%

\usepackage{amsfonts,graphicx}
%\usepackage{amsmath}
\usepackage{algorithm}
%\usepackage[noend]{algpseudocode}
\usepackage{verbatim}

\usepackage[T1]{fontenc}
\usepackage[utf8]{inputenc}
\usepackage[english]{babel}
%\usepackage[margin=1in]{geometry}
\usepackage{natbib}
%\setcitestyle{nocompress}
%\bibpunct{(}{)}{;}{a}{}{,}

% \usepackage[nocompress]{cite}
%\usepackage{graphicx}
%\usepackage[cmex10]{amsmath}
%\usepackage{authblk}
% \usepackage[colorinlistoftodos,prependcaption,textsize=tiny]{todonotes}
\usepackage{macros}
\usepackage{times}


%
% The following commands set up the lecnum (lecture number)
% counter and make various numbering schemes work relative
% to the lecture number.
%
\newcounter{lecnum}
\renewcommand{\thepage}{\arabic{page}}
\renewcommand{\thesection}{\arabic{section}}
\renewcommand{\theequation}{\arabic{equation}}
\renewcommand{\thefigure}{\arabic{figure}}
\renewcommand{\thetable}{\arabic{table}}

%
% The following macro is used to generate the header.
%
\newcommand{\lecture}[4]{
   \pagestyle{myheadings}
   \thispagestyle{plain}
   \newpage
   \setcounter{lecnum}{#1}
   \setcounter{page}{1}
   \noindent
   \begin{center}
   \framebox{
      \vbox{\vspace{2mm}
    %\hbox to 6.28in { {\bf Report on my work with Adobe
		%\hfill Spring 2018} }
       \vspace{4mm}
       \hbox to 6.28in { {\Large \hfill Stochastic Low-Rank Latent Bandits \hfill} }
       \vspace{2mm}
       \hbox to 6.28in { {\it Adobe Advisor(s): Branislav Kveton, Anup Rao, Zheng Wen \hfill Intern: Subhojyoti Mukherjee} }
      \vspace{2mm}}
   }
   \end{center}
   %\markboth{Lecture #1: #2}{Lecture #1: #2}

   {\bf Disclaimer}: {\it These notes have not been subjected to the
   usual scrutiny reserved for formal publications.  They may be distributed
   outside this communication only with the permission of the Advisor(s).}
   \vspace*{4mm}
}
%
% Convention for citations is authors' initials followed by the year.
% For example, to cite a paper by Leighton and Maggs you would type
% \cite{LM89}, and to cite a paper by Strassen you would type \cite{S69}.
% (To avoid bibliography problems, for now we redefine the \cite command.)
% Also commands that create a suitable format for the reference list.
\renewcommand{\cite}[1]{[#1]}
\def\beginrefs{\begin{list}%
        {[\arabic{equation}]}{\usecounter{equation}
         \setlength{\leftmargin}{2.0truecm}\setlength{\labelsep}{0.4truecm}%
         \setlength{\labelwidth}{1.6truecm}}}
\def\endrefs{\end{list}}
\def\bibentry#1{\item[\hbox{[#1]}]}

%Use this command for a figure; it puts a figure in wherever you want it.
%usage: \fig{NUMBER}{SPACE-IN-INCHES}{CAPTION}
\newcommand{\fig}[3]{
			\vspace{#2}
			\begin{center}
			Figure #1:~#3
			\end{center}
	}
% Use these for theorems, lemmas, proofs, etc.
%\newtheorem{theorem}{Theorem}
%\newtheorem{lemma}[theorem]{Lemma}
%\newtheorem{proposition}[theorem]{Proposition}
%\newtheorem{claim}[theorem]{Claim}
%\newtheorem{corollary}[theorem]{Corollary}
%\newtheorem{definition}[theorem]{Definition}
%\newenvironment{proof}{{\bf Proof:}}{\hfill\rule{2mm}{2mm}}

% **** IF YOU WANT TO DEFINE ADDITIONAL MACROS FOR YOURSELF, PUT THEM HERE:

%\newcommand\E{\mathbb{E}}


\begin{document}
%FILL IN THE RIGHT INFO.
%\lecture{**LECTURE-NUMBER**}{**DATE**}{**LECTURER**}{**SCRIBE**}
\lecture{1}{A - Title}{Lecturer Name}{scribe-name}
%\footnotetext{These notes are partially based on those of Nigel Mansell.}

% **** YOUR NOTES GO HERE:

% Some general latex examples and examples making use of the
% macros follow.  
%**** IN GENERAL, BE BRIEF. LONG SCRIBE NOTES, NO MATTER HOW WELL WRITTEN,
%**** ARE NEVER READ BY ANYBODY.
%This lecture's notes illustrate some uses of
%various \LaTeX\ macros.  
%Take a look at this and imitate.

\begin{abstract}
To be written.
\end{abstract}


\section{Introduction}
\label{intro}
In this paper, we study the problem of recommending the best items to users who are coming sequentially. The learner has access to very less prior information about the users and it has to adapt quickly to the user preferences and suggest the best item to each user. Furthermore, we consider the setting where users are grouped into clusters and within each cluster the users have the same choice of the best item, even though their quality of preference may be different for the best item. These clusters along with the choice of the best item for each user are unknown to the learner.  Also, we assume that each user has a single best item preference.

	This complex problem can be conceptualized as a low rank stochastic bandit problem where there are $K$ users and $L$ items. The reward matrix, denoted by $\bar{R}\in [0,1]^{K\times L}$,  generating the rewards for user, item pair has a low rank structure. The online learning game proceeds as follows, at every timestep $t$,  nature reveals one user (or row) from $\bar{R}$ where user is denoted by $i_t$. The learner selects one item (or column) from $\bar{R}$, where the item is denoted by $j_t$. Then the learner receives one noisy feedback $r_{t}(i_t,j_t)\sim\mathcal{N}(\bar{R}(i_t,j_t)),\sigma^2)$, where $\mathcal{N}$ is a distribution over the entries in $\bar{R}$, $\sigma^2$ is variance and $\E[r_{t}(i_t,j_t)] = \bar{R}(i_t,j_t)$. Then the goal of the learner is to minimize the cumulative regret by quickly identifying the best item $j^*_t$ for each $i_t\in \bar{R}$ where $\bar{R}_{i_t,j^*_t} = \argmax _{j\in[L]}\lbrace \bar{R}_{i_t,j} \rbrace$. %Also, there exist $C$ clusters, indexed from $\lbrace 1,2,\ldots, C\rbrace \in\C$ such that for each $i_t\in U$, 


\subsection{Notations, Problem Formulation and Assumptions}
%Extending the discussion in the previous section, we propose a Generalized Latent Bandit model where each $v_{a,b}$ can be considered as a mixture of several $\lbrace v_{a,c}\rbrace_{c\in\C}$. This is a harder problem than Latent Bandit model because now the rows in each cluster are not exactly identical but the index of the optimal arm (column) is similar in each row.
	
	We define $[n] = \lbrace 1,2,\ldots, n\rbrace$ and for any two sets $A$ and $B$, $A^B$ denotes the set of all vectors who take values from $A$ and are indexed by $B$. Let, $R\in [0,1]^{K\times L}$ denote any matrix, then $R(I,:)$ denote any submatrix of $k$ rows such that $I\in[K]^k$ and similarly $R(:,J)$ denote any submatrix of $j$ columns such that $J\in[L]^{j}$.
	
	Let $\bar{R}$ be reward matrix of  dimension $K\times L$ where $K$ is the number of user or rows and $L$ is the number of arms or columns. Also, let us assume that this matrix  $\bar{R}$ has a low rank structure of rank $d << \min\lbrace L,K\rbrace$. Let $U$ and $V$ denote the latent matrices for the users and items, which are not visible to the learner such that,
\begin{align*}
	\bar{R} = UV^{\intercal} \textbf{ \hspace*{4mm}   s.t.   \hspace*{4mm}} U\in [ \mathbb{R}^+ ]^{K\times d} \textbf{, } V\in  [0,1]^{L\times d} 
\end{align*}	  
	
	Furthermore, we put a constraint on $V$ such that, $\forall j\in [L]$, $ \norm{V(j,:)}_1 \leq 1$. 
	
	
\begin{assumption}
\label{assm:1}
We assume that there exists $d$-column base factors, denoted by $V(J^*,:)$, such that all rows of $V$ can be written as a convex combination of $V(J^*,:)$ and the zero vector and $J^* = [d]$. We denote the column factors by $V^* = V(J*,:)$. Therefore, for any $i\in [L]$, it can be represented by
\begin{align*}
V(i,:) = a_i V(J^*,:) , 
\end{align*}
where $\exists a_i\in [0,1]^{d}$ and $ \norm{a_i}_1 \leq 1$.
\end{assumption}

%In this paper, in addition to the noisy setting explained in section \ref{intro} we first analyze the proposed algorithm in the easier noise free setting. In the noise free setting, the nature reveals the row $i_t$, and when the learner selects the column $j_t$, it observes the mean of the distribution $\bar{R}(i_t,j_t)$.

%\begin{assumption}
%\label{assm:round-robin}
%We assume that nature is revealing the user $i$ in $\bar{R}(i,:), \forall i\in [K]$  in a Round-Robin fashion such that at timestep $t$, nature reveals $i_t = (t \mod K) + 1$.
%\end{assumption}

\begin{assumption}
\label{assm:d-items}
For each user $i_t$ revealed by the nature at round $t$, the learner is allowed to select atmost $d$-items, where $d$ is the rank of the matrix $\bar{R}$.
\end{assumption}

The above assumption \ref{assm:d-items} can be conceptualized in this real-world scenario where the learner has  to suggest movies to users and each movie belongs to a different genre (say thriller, romance, comedy, etc). So, the learner can suggest $d$ movies belonging to different genres to each user, and the user can click one, or both, or none of the recommended movies.


The main goal of the learning agent is to minimize the cumulative regret until the end of horizon $n$. We define the cumulative regret, denoted by $\mathcal{R}_n$ as,

\begin{align*}
\mathcal{R}_n = \sum_{t=1}^{n}\bigg\lbrace \sum_{z=1}^{d} \bigg( r_{z,t}\left(i_{t}, j^*_{z,t} \right) - r_{z, t}\left( i_{t}, j_{z,t}\right)\bigg)\bigg\rbrace
\end{align*}

where, $j^*_{z,t} = \argmax_{j\in [L]}\lbrace \bar{R}(i_t,j)\rbrace$ and $j_{z,t}$ be the suggestion of the learner for the $i_t$ -th user for  $z=1,2,\ldots, d$. Note that $r_{t}\left(i_t, j^*_{z,t} \right)\sim \mathcal{N}(\bar{R}\left(i_t, j^*_{z,t} \right),\sigma^2)$ and $r_{t}\left(i_t, j_{z,t} \right)\sim \mathcal{N}(\bar{R}\left(i_t, j_{z,t} \right), \sigma^2)$. Taking expectation over both sides, we can show that,

\begin{align*}
\E[\mathcal{R}_n] = \E\left [ \sum_{t=1}^{n}\bigg\lbrace\sum_{z=1}^{d} \bigg( r_{z,t}\left(i_t, j^*_{z,t} \right) - r_{z, t}\left( i_t, j_{z, t}\right)\bigg)\bigg\rbrace\right] = \E\left [ \sum_{t=1}^{T} \sum_{z=1}^{d} \bigg( N_{i_t,j_{z,t}}\bigg) \right ]\Delta_{i_t,j_{z,t}}
\end{align*}

where, $\Delta_{i_t,j_{z,t}} = \bar{R}(i_t,j^*_{z,t}) - \bar{R}(i_t,j_{z,t})$ and $N_{i_t,j_{z,t}}$ is the number of times the learner has observed the $j_{z,t}$-th item for the $i_t$-th user for $z=1,2,\ldots, d$. Let, $\Delta = \min_{i\in[K],j\in[L]}\lbrace \Delta_{i,j}\rbrace$ be the minimum gap over all the user, item pair in $\bar{R}$.


%\subsection{Notations and Assumptions}
%\input{notations}
	
\subsection{Related Works}
\input{related}
	

\section{Contributions}
\input{contributions}

\newpage
\section{Proposed Algorithms}
%\subsection{Noise-Free Setting}


%\begin{assumption}
%\label{assm:d-indep}
%We assume that any $d$ sets of rows or columns are independent in $\bar{R}$.
%\end{assumption}

\algblock{ArmElim}{EndArmElim}
\algnewcommand\algorithmicArmElim{\textbf{\em Arm Elimination}}
 \algnewcommand\algorithmicendArmElim{}
\algrenewtext{ArmElim}[1]{\algorithmicArmElim\ #1}
\algrenewtext{EndArmElim}{\algorithmicendArmElim}

\algblock{ResParam}{EndResParam}
\algnewcommand\algorithmicResParam{\textbf{\em Reset Parameters}}
 \algnewcommand\algorithmicendResParam{}
\algrenewtext{ResParam}[1]{\algorithmicResParam\ #1}
\algrenewtext{EndResParam}{\algorithmicendResParam}

\algblock{ColRec}{EndColRec}
\algnewcommand\algorithmicColRec{\textbf{\em Column Reconstruction}}
 \algnewcommand\algorithmicendColRec{}
\algrenewtext{ColRec}[1]{\algorithmicColRec\ #1}
\algrenewtext{EndColRec}{\algorithmicendColRec}

\algblock{ColElim}{EndColElim}
\algnewcommand\algorithmicColElim{\textbf{\em Structured Column Elimination}}
 \algnewcommand\algorithmicendColElim{}
\algrenewtext{ColElim}[1]{\algorithmicColElim\ #1}
\algrenewtext{EndColElim}{\algorithmicendColElim}

\algblock{ColElimU}{EndColElimU}
\algnewcommand\algorithmicColElimU{\textbf{\em User Column Elimination}}
 \algnewcommand\algorithmicendColElimU{}
\algrenewtext{ColElimU}[1]{\algorithmicColElimU\ #1}
\algrenewtext{EndColElimU}{\algorithmicendColElimU}

\algblock{Explore}{EndExplore}
\algnewcommand\algorithmicExplore{\textbf{\em Explore}}
 \algnewcommand\algorithmicendExplore{}
\algrenewtext{Explore}[1]{\algorithmicExplore\ #1}
\algrenewtext{EndExplore}{\algorithmicendExplore}


\algblock{Exploit}{EndExploit}
\algnewcommand\algorithmicExploit{\textbf{\em Exploit}}
 \algnewcommand\algorithmicendExploit{}
\algrenewtext{Exploit}[1]{\algorithmicExploit\ #1}
\algrenewtext{EndExploit}{\algorithmicendExploit}

\algnewcommand\algorithmicforeach{\textbf{for each}}
\algdef{S}[FOR]{ForEach}[1]{\algorithmicforeach\ #1\ \algorithmicdo}

%\newcommand\Algphase[1]{%
%\vspace*{-.7\baselineskip}\Statex\hspace*{\dimexpr-\algorithmicindent-2pt\relax}\rule{\textwidth}{0.4pt}%
%\Statex\hspace*{-\algorithmicindent}\textbf{#1}%
%\vspace*{-.7\baselineskip}\Statex\hspace*{\dimexpr-\algorithmicindent-2pt\relax}\rule{\textwidth}{0.4pt}%
%}

We propose two algorithms, one each for noise free and noisy setting. These algorithms are based on UCB-Improved, which is an arm elimination algorithm from \citet{auer2010ucb} and is suitable for the stochastic bandit setting. Both these algorithms are phase based column (arm) elimination algorithms where in each phase we select all the surviving columns equal number of times  and then  eliminate some sub-optimal columns based on some elimination criteria. 

The algorithms are initialized with the estimate $\hat{R}_{i,j}=0, \forall i\in[K], j\in[L]$. In the $m$-th phase, we denote the set of surviving columns as $\B_m$, the set of explored columns as $\Z_m$. The rows (users) are divided into equivalence classes which are contained in $\C$. We denote each equivalence class as $\G_b$, where $b$ is indexed from $1,2,\ldots , \frac{K}{\gamma}$ and these class are contained in $\C$.  $N_m$ denotes the phase length for the $m$-th phase and each phase length consist of $\gamma |B_m| n_m $, where $ \gamma = \lceil\sqrt{K} \rceil $ is the exploration parameter and $ n_m $ is the number of times we select each surviving columns. In the column elimination sub-module we eliminate a sub-optimal column by making sure that it is not one of the $d$-best columns. In the reset parameters sub-module, we increase the exploration bonus for the next phase so that more exploration is conducted for the surviving columns. Note, that for these two sub-modules, the noise free and noisy setting has two different approaches which will be explained in subsection \ref{alg:noisefree} and \ref{alg:noisy} respectively. Finally, if the algorithm has eliminated $L-d$ columns, it fully explores the $d$ best columns (in the noise free setting) and for all the users it always selects the column $j^*_t$, where $j_t^* \leftarrow \argmax_{j\in[\B_m]} {\hat{R}(i_t,j)}$. Whereas, in the noisy setting, the GLB-UCB runs the UCB1 algorithm for the remaining $d$ columns.

\subsection{Noise Free Setting}
\label{alg:noisefree}

In the noise free setting, since at every pull of a column $j$ for a user $i$, the algorithm observes the expected reward $\hat{R}_{i,j} = \bar{R}_{i,j}$, so $n_0 =  n_m = 1, \forall m$. Furthermore, we do not require any confidence interval in the noise free setting for column elimination sub-module. The pseudo-code of this is shown in Algorithm \ref{alg:NFGLB}.

\begin{algorithm}[!th]
\caption{Noise-Free GLBUCB}
\label{alg:NFGLB}
\begin{algorithmic}[1]
\State {\bf Input:} Time horizon $T$, $Rank(\bar{R}) = d$.
\State {\bf Explore Parameter:} $\gamma = \lceil\sqrt{K} \rceil$.
\State {\bf Initialization:} $\hat{R}(i,j) \leftarrow 0, \forall i\in [K], j\in [L]$, $m=0$, $\B_0 \leftarrow \A$, $\Z_0 \leftarrow \emptyset$, $\C \leftarrow \emptyset$, $T[j]\leftarrow 0, \forall j\in [L]$, $i_0=1$, $j_0=1$ , $n_0 = 1 $, $N'_0 =  \gamma |\B_0|n_0 $ and $N_0 = N'_0 + K n_0 d$.
%\State Create set $\C$ of equivalence classes such that 
\ForEach{$b\in \left[0,\frac{K}{\gamma}-1\right]$}  \Comment{Create equivalence class $\C$}
\State  $\G_{b+1}\leftarrow \left\lbrace i\in \left[\frac{bK}{\gamma}+1,\frac{(b+1)K}{\gamma}\right]\right\rbrace$ and $\C\leftarrow \C\cup \G_{b+1}$.
\EndFor
\Procedure{ChequerBoardExploration}{$\gamma,\B_m,\Z_m$}\Comment{Conduct ChequerBoard Exploration}
\If{$i_0 \leq \gamma $}
\State Choose $j_0\in\B_m\setminus\Z_m$, observe $\bar{R}(i_0,j_0)$ and $\hat{R}(i_0,j_0)\leftarrow \bar{R}(i_0,j_0)$.
\State $i_0 \leftarrow i_0 + 1$.
\Else
\State $\Z_m \leftarrow \Z_m \cup \lbrace j_0\rbrace$
\State Choose $j_0\in\B_m\setminus \Z_m$ uniform randomly and $i_0 \leftarrow 1$.
\EndIf
\EndProcedure

\Procedure{PhaseExploreExploit}{$i_t,\B_m,\mathcal{V}_{m}$} \Comment{Conduct Phase Explore-Exploit
\State Select $j_0\in \argmax_{j\in\B_m\cap\mathcal{V}_{i_t,m}}\left\lbrace \hat{R}(i_t,j)\right\rbrace$

%\State \ForEach{$ j\in\B_m$}
%\State $s \leftarrow 0$
%\State \ForEach{$i\in [K]$}
%\State \If{$\hat{R}(i,j) >  \hat{R}(i,j'), \forall j'\in\B_{m}\setminus j $}
%\State $s \leftarrow s + 1$
%\EndIf
%\EndFor
%\State $T[j] \leftarrow T[j] + s$
%\EndFor
%\State $\S_m \leftarrow j \in \max_{d}\lbrace T\rbrace$, set $T[j]\leftarrow 0, \forall j\in [L]$.
%\State \ForEach{$j \in\S_m$}
%\State Choose $j, \forall i \in  [K]$, observe $\bar{R}(i,j)$ and Update $\hat{R}(i,j)\leftarrow \bar{R}(i_t,,j)$
%\EndFor
\EndProcedure

\Procedure{GLB-UCB}{}\Comment{Start GLB-UCB}
\For{$t=1,..,T$}	
\State Nature reveals $i_t$ such that $i_t \leftarrow (t \mod K) + 1$ (Round-Robin).
\If{$|\B_m| > d$}
\If{$t \leq N_m$ and $t \leq N'_{m}$}  
\State Call ChequerBoard Exploration
\ElsIf{$t \leq N_m$ and $t > N'_{m}$}  
\State Call Phase Exploit
\Else
\ColElim
\State \ForEach{$\G_b\in\C$}
\State \While{$\exists j\in\B_m \textbf{ such that } {\forall i\in \G_b: \hat{R}(i,j) < \max_{j'\in\B_{m}\setminus j}\lbrace\hat{R}(i,j') \rbrace}$}
\State $\B_m \leftarrow \B_m \setminus \lbrace j \rbrace$. 
\EndWhile
\EndFor
\EndColElim
\ColElimU
\State \ForEach{$i\in [K]$}
\State \ForEach{$j'\in\B_m\cap\mathcal{V}_{i,m}$}
\State \If{$\hat{R}(i,j') < \max_{j\in\B_m\cap\mathcal{V}_{i,m}}\lbrace \hat{R}(i,j)\rbrace$}
\State $\mathcal{V}_{i,m}\leftarrow \mathcal{V}_{i,m} \cup \lbrace j'\rbrace$
\EndIf
\EndFor
\EndFor
\State
\EndColElimU
\ResParam
\State $N'_{m + 1}  \leftarrow t +  \gamma |\B_m| n_0 $
\State $N_{m+1} \leftarrow N'_{m + 1} + K n_0 d$ and $m \leftarrow m + 1$.
\EndResParam
\EndIf
\Else \textbf{ (Full Exploit) }
%\State  Explore all $j\inB_{m}$ fully
\State  Select column $j_t^*$, observe $\bar{R}(i_t,j_t^*)$ where $j_t^* \leftarrow \argmax_{j\in[\B_m]} {\hat{R}(i_t,j)}$ and $\hat{R}(i_t,,j_t^*)\leftarrow \bar{R}(i_t,,j_t^*)$.
\EndIf
\EndFor
\EndProcedure
\end{algorithmic}
%\vspace*{-0.42em}
\end{algorithm}


\subsection{Noisy setting}
\label{alg:noisy}

In the noisy setting, at every pull the algorithm observes a random reward which is drawn from the distribution $\N(\bar{R}_{i,j}, \sigma^2)$, so the algorithm samples each column more number of times ($n_m$) in each phase and increases the exploration from phase to phase. Moreover, in the column elimination sub-module, the confidence interval $U_m(\epsilon_m, n_m)$  helps in eliminating a sub-optimal column with high probability in the noisy setting. The pseudo-code of this is shown in Algorithm \ref{alg:NGLB}.


\begin{algorithm}[!th]
\caption{Noisy GLB-UCB}
\label{alg:NGLB}
\begin{algorithmic}[1]
\State {\bf Input:} Time horizon $T$, $Rank(\bar{R}) = d$.
\State {\bf Explore Parameter:} $\gamma = \lceil\sqrt{K} \rceil$, $\psi = T$.
\State {\bf Definition:} $U_m(\epsilon_m, n_m) = \sqrt{\dfrac{\psi\log(T\epsilon_m^2)}{2n_m} }$
\State {\bf Initialization:} $\forall i\in [K], j\in [L], \hat{R}(i,j) \leftarrow 0$, $m=0$, $\B_0 \leftarrow \A$, $\Z_0 \leftarrow \emptyset$, $\C \leftarrow \emptyset$,  $i_0=1$, $j_0=1$ , $\epsilon_0 = 1$, $M=\dfrac{1}{2}\log_2\left( \frac{T}{e}\right)$, $n_0 = \dfrac{2\log(\psi T\epsilon_{0}^2)}{\epsilon_{0}} $, $N'_0 = K n_0 d$ and $N_0 = N'_0 + \gamma |\B_0|n_0$.
%\State Create set $\C$ of equivalence classes such that 
\ForEach{$b\in \left[0,\frac{K}{\gamma}-1\right]$} (Create equivalence class $\C$)
\State  $\G_{b+1}\leftarrow \left\lbrace i\in \left[\frac{bK}{\gamma}+1,\frac{(b+1)K}{\gamma}\right]\right\rbrace$ and $\C\leftarrow \C\cup \G_{b+1}$.
\EndFor
\For{$t=1,..,T$}	
\State Nature reveals $i_t$ such that $i_t \leftarrow (t \mod K) + 1$ (Round-Robin).
\If{$|\B_m| > d$} 
\If{$t \leq N_m$ and $t \leq N'_m$} \textbf{ (Phase Exploit) }
\State Choose $j_0\leftarrow \argmax_{j\in_{B_m}}\left\lbrace \hat{R}(i_t,j)  + \sqrt{\dfrac{2\log T}{n_{i_t,j}}}\right\rbrace $, observe $r_t(i_t,j_0)\sim \mathcal{N}(\bar{R}(i_t,j_0),\sigma^2)$ and $\hat{R}(i_t,j_0)\leftarrow \dfrac{\sum_{s=1}^{t}r_s(i_t,j_0)[\mathbb{I}_{s,i_t} = j_0]}{n_{i_t,j_0}}$.
\ElsIf{$t \leq N_m$ and $t > Nx_m$} \textbf{ (Phase Explore) }
\If{$i_0 \leq \gamma $}
\State Choose $j_0$, observe $r_t(i_0,j_0)\sim\mathcal{N}(\bar{R}(i_0,j_0),\sigma^2) $ and $\hat{R}(i_0,j_0)\leftarrow \dfrac{\sum_{s=1}^{t}r_s(i_0,j_0)[\mathbb{I}_{s,i_0} = j_0]}{n_{i_0,j_0}}$.
\State $i_0 \leftarrow i_0 + 1$.
\Else
\State $\Z_m \leftarrow \Z_m \cup \lbrace j_0\rbrace$
\State Choose $j_0\in\B_m\setminus \Z_m$ uniform randomly and $i_0 \leftarrow 1$.
\EndIf
\Else
\ColElim
\State \ForEach{$\G_b\in\C$}
\State \While{$\exists j\in\B_m \textbf{ such that } {\forall i\in \G_b: \hat{R}(i,j) + U_m(\epsilon_m, n_m) < \max_{j'\in\B_{m}\setminus j}\lbrace\hat{R}(i,j') - U_m(\epsilon_m, n_m)\rbrace}$}
\State $\B_m \leftarrow \B_m \setminus \lbrace j \rbrace$. 
%\EndIf
\EndWhile
\EndFor
\EndColElim
\ResParam
\State $\epsilon_{m+1} \leftarrow \dfrac{\epsilon_m}{2}$ and $n_{m+1} \leftarrow \dfrac{2\log(\psi T\epsilon_{m+1}^2)}{\epsilon_{m+1}}$
\State $N'_{m+1}\leftarrow t + K n_{m+1} d$
\State $N_{m+1} \leftarrow N'_{m+1} + \gamma |\B_m|n_{m+1}$ and $m \leftarrow m + 1$.
\EndResParam
\EndIf
\Else \textbf{ (Full Exploit) }
%\Select Explore all $j\inB_{m}$ fully.
\State  Select column $j_t^*$, observe $r_t(i_t,j_t^*)\sim \mathcal{N}(\bar{R}(i_t,j_t^*),\sigma^2)$ where $j_t^* \leftarrow \argmax_{j\in[\B_m]} \left\lbrace \hat{R}(i_t,j) + \sqrt{\dfrac{2\log T}{n_{i_t,j}}}\right\rbrace$ and $\hat{R}(i_t,,j_t^*)\leftarrow \dfrac{\sum_{s=1}^{t}r_s(i_t,j_t^*)[\mathbb{I}_{s,i_t} = j_t^*]}{n_{i_tj_t^*}}$.
\EndIf
\EndFor
\end{algorithmic}
%\vspace*{-0.42em}
\end{algorithm}






\section{Main Results}
\input{results}



\section{Proofs}
%\begin{proof}
%Considering any arbitrary row $i\in [K]$, we can show that,
%\begin{align*}
%\argmax_{j\in[L]} U(i,:){V(j,:)}^{\intercal}  &= U(i,:)V(j^*(i),:)^{\intercal}\\
%& \overset{(a)}{=} U(i,:)\left(a_{j^*(i)}V(J^*,:)\right)^{\intercal}\\
%& = \sum_{k=1}^{d}a_{j^*(i)}(k)U(i,:)V(j^*(i),:)^{\intercal}\\
%& \leq \argmax a_{j^*(i)}(k)U(i,:)V(k,:)^{\intercal} \\
%& \leq \argmax_{k\in[d]}U(i,:){V(k,:)}^{\intercal}   ,
%\end{align*}
%where $(a)$ is from Assumption \ref{assm:1}.
%\end{proof}

\subsection{Proof of Regret Bound for Noise Free case}

\begin{proof}
\textbf{Step 1. (Some notations):} We denote the number of times we observe $r_t\left( i ,j \right) \in \bar{R}\left( i  ,j \right)$ till $t$-th timestep as $z_t\left( i,j\right)$. Let, $\J^*$ be the set of $d$ best columns.

\textbf{Step 2. (Assumptions): } We assume that $\bar{R}(i,j)$ is a square matrix of dimension $K\times K$, where $K \geq 4$ and $\lceil\sqrt{K} \rceil \in \mathbb{Z}^+$.

\textbf{Step 3. (Total pulls in phase $0$)}: The total number of pulls in phase $0$ is accrued due to \textbf{Chequerboard exploration} and \textbf{Best-d exploration}, which gives,
\begin{align*}
Kd n_0 + \gamma Kn_0 & \overset{(a)}{=} Kd + K\lceil\sqrt{K} \rceil\\
&= Kd + K^{\frac{3}{2}},
\end{align*}

where $(a)$ is obtained because $n_0 = 1$ and $\gamma = \lceil\sqrt{K} \rceil$.

\textbf{Step 4. (Probability of $j\in B_m$ and $j\notin \J^*$ after column-equivalence domination):} In the worst case the chequerboard exploration only observes $j\in J^*$ over the same equivalence class $E_1\in \C$ (say). Hence, the number of arms $j'\in \B_m$ surviving after the $m$-th phase cannot be more than,

\begin{align*}
K - \dfrac{\sqrt{K}}{d} = \sqrt{K}\left(\sqrt{K} - \dfrac{1}{d}\right).
\end{align*} 

\textbf{Step 5. (Probability of  $j\in \S_m$ and $j\notin \J^*$):} 

\end{proof}



\newpage
\section{Experiments}
\begin{figure}[!th]
    \centering
    \begin{tabular}{cc}
    \setlength{\tabcolsep}{0.1pt}
    \subfigure[0.25\textwidth][Expt-$1$: $3$ Users, $3$ Bernoulli-distributed arms and Round-Robin]
    %with $r_{i_{{i}\neq {*}}}=0.07$ and $r^{*}=0.1$
    {
    		\pgfplotsset{
		tick label style={font=\Large},
		label style={font=\Large},
		legend style={font=\Large},
		ylabel style={yshift=5pt},
		%legend style={legendshift=32pt},
		}
        \begin{tikzpicture}[scale=0.8]
      	\begin{axis}[
		xlabel={timestep},
		ylabel={Cumulative Regret},
		grid=major,
        %clip mode=individual,grid,grid style={gray!30},
        clip=true,
        %clip mode=individual,grid,grid style={gray!30},
  		legend style={at={(0.5,1.4)},anchor=north, legend columns=3} ]
      	% UCB
		\addplot table{results/NewExpt/Expt2/comp_subsampled_CustomUCB0RR1.txt};
		\addplot table{results/NewExpt/Expt2/comp_subsampled_CustomEXP30RR1.txt};
      	\legend{UCB1-RR,EXP3-RR}      	
      	\end{axis}
      	\end{tikzpicture}
  		\label{fig:1}
    }
    &
    \subfigure[0.25\textwidth][Expt-$2$: $3$ Users, $3$ Bernoulli-distributed arms and Uniform Sampling ]
    %with $r_{i_{{i}\neq {*}}}=0.07$ and $r^{*}=0.1$
    {
    		\pgfplotsset{
		tick label style={font=\Large},
		label style={font=\Large},
		legend style={font=\Large},
		ylabel style={yshift=5pt},
		%legend style={legendshift=32pt},
		}
        \begin{tikzpicture}[scale=0.8]
      	\begin{axis}[
		xlabel={timestep},
		ylabel={Cumulative Regret},
		grid=major,
        %clip mode=individual,grid,grid style={gray!30},
        clip=true,
        %clip mode=individual,grid,grid style={gray!30},
  		legend style={at={(0.5,1.4)},anchor=north, legend columns=3} ]
      	% UCB
		\addplot table{results/NewExpt/Expt2/comp_subsampled_CustomUCB0US1.txt};
		\addplot table{results/NewExpt/Expt2/comp_subsampled_CustomEXP30US1.txt};
      	\legend{UCB1-US,EXP3-US}      	
      	\end{axis}
      	\end{tikzpicture}
  		\label{fig:2}
    }
    \\
    \subfigure[0.25\textwidth][Expt-$2$: $20$ Users, $3$ Bernoulli-distributed arms, Round-Robin, Noise-Free Setting ]
    %with $r_{i_{{i}\neq {*}}}=0.07$ and $r^{*}=0.1$
    {
    		\pgfplotsset{
		tick label style={font=\Large},
		label style={font=\Large},
		legend style={font=\Large},
		ylabel style={yshift=5pt},
		%legend style={legendshift=32pt},
		}
        \begin{tikzpicture}[scale=0.8]
      	\begin{axis}[
		xlabel={timestep},
		ylabel={Cumulative Regret},
		grid=major,
        %clip mode=individual,grid,grid style={gray!30},
        clip=true,
        %clip mode=individual,grid,grid style={gray!30},
  		legend style={at={(0.5,1.4)},anchor=north, legend columns=3} ]
      	% UCB
		\addplot table{results/NewExpt/Expt3/comp_subsampled_CustomUCB0RR.txt};
		\addplot table{results/NewExpt/Expt3/comp_subsampled_GLBUCB0RR.txt};
		\addplot table{results/NewExpt/Expt3/comp_subsampled_CustomEXP30RR.txt};
      	\legend{UCB1-RR,GLBUCB-RR, EXP3-RR}      	
      	\end{axis}
      	\end{tikzpicture}
  		\label{fig:3}
    }
    \end{tabular}
    \caption{A comparison of the cumulative regret incurred by the various bandit algorithms. }
    \label{fig:karmed}
    \vspace*{-1em}
\end{figure}



\section{Conclusions and Future Direction}
\input{conclusions}


\clearpage
\newpage
%\bibliographystyle{plainnat}
\bibliographystyle{apalike}
\bibliography{biblio}

\end{document}





