In \citet{maillard2014latent} the authors propose the Latent Bandit model where there are two sets: 1) set of arms denoted by $\A$ and 2) set of types denoted by $\B$ which contains the latent information regarding the arms. The latent information for the arms are modeled such that the set $\B$ is assumed to be partitioned into $|C|$ clusters, indexed by $\B_1, \B_2, \ldots, \B_C \in \C$ such that the distribution $v_{a,b}, a\in\A, b\in\B_c$ across each cluster is same.  Note, that the identity of the cluster is unknown to the learner. At every timestep $t$, nature selects a type $b_t\in\B_c$ and then the learner selects an arm $a_t\in\A$ and observes a reward $r_{t}(a,b)$ from the distribution $v_{a,b}$.
	
	Another way to look at this problem is to imagine a matrix of dimension $|A|\times |B|$ where again the rows in $\B$ can be partitioned into $|C|$ clusters, such that the distribution across each of this clusters are same. Now, at every timestep $t$ one of this row is revealed to the learner and it chooses one column such that the $v_{a,b}$ is one of the $\lbrace v_{a,c}\rbrace_{c\in\C}$ and the reward for that arm and the user is revealed to the learner.
	
	This is actually a much simpler approach than the setting we considered  because note that the distributions across each of the clusters $\lbrace v_{a,c}\rbrace_{c\in\C}$ are identical and estimating one cluster distribution will reveal all the information of the users in each cluster.