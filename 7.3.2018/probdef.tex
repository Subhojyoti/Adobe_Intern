%Extending the discussion in the previous section, we propose a Generalized Latent Bandit model where each $v_{a,b}$ can be considered as a mixture of several $\lbrace v_{a,c}\rbrace_{c\in\C}$. This is a harder problem than Latent Bandit model because now the rows in each cluster are not exactly identical but the index of the optimal arm (column) is similar in each row.
	
	We define $[n] = \lbrace 1,2,\ldots, n\rbrace$ and for any two sets $A$ and $B$, $A^B$ denotes the set of all vectors who take values from $A$ and are indexed by $B$. Let, $R\in [0,1]^{K\times L}$ denote any matrix, then $R(I,:)$ denote any submatrix of $k$ rows such that $I\in[K]^k$ and similarly $R(:,J)$ denote any submatrix of $j$ columns such that $J\in[L]^{j}$.
	
	Let $\bar{R}$ be reward matrix of  dimension $K\times L$ where $K$ is the number of user or rows and $L$ is the number of arms or columns. Also, let us assume that this matrix  $\bar{R}$ has a low rank structure of rank $d << \min\lbrace L,K\rbrace$. Let $U$ and $V$ denote the latent matrices for the users and items, which are not visible to the learner such that,
\begin{align*}
	\bar{R} = UV^{\intercal} \textbf{ \hspace*{4mm}   s.t.   \hspace*{4mm}} U\in [ \mathbb{R}^+ ]^{K\times d} \textbf{, } V\in  [0,1]^{L\times d} 
\end{align*}	  
	
	Furthermore, we put a constraint on $V$ such that, $\forall j\in [L]$, $ ||V(j,:)||_1 \leq 1$. 
	
	%Also, we assume that $R$ is a hott topics matrix such that there exist $d$-base row factors denoted by $U(\mathbb{I}^*,:)$ such that all rows in $U$ can be expressed as a convex combination of the rows of $U(\mathbb{I},:)$ and the zero vector. 
	
%	\textbf{Comment 1: The constraints on $V$ needs to be worked out as pointed out by Anup.}
%	
%	\textbf{Comment 2: Only row base factors are required. Why??}


\begin{assumption}
\label{assm:1}
We assume that there exists $d$-column base factors, denoted by $V(J^*,:)$, such that all rows of $V$ can be written as a convex combination of $V(J^*,:)$ and the zero vector and $J^* = [d]$. We denote the column factors by $V^* = V(J*,:)$. Therefore, for any $i\in [L]$, it can be represented by
\begin{align*}
V(i,:) = a_i V(J^*,:) , 
\end{align*}
where $\exists a_i\in [0,1]^{d}$ and $ ||a_i||_1 \leq 1$.
\end{assumption}

%In the noise free setting, in addition to Assumption \ref{assm:1} we assume two further assumptions.

\begin{assumption}
\label{assm:round-robin}
We assume that nature is revealing $i$ in $\bar{R}(i,:), \forall i\in [K]$  in a Round-Robin fashion.
\end{assumption}



%\begin{assumption}
%
%\end{assumption}