%
% This is the LaTeX template file for lecture notes for CS267,
% Applications of Parallel Computing.  When preparing 
% LaTeX notes for this class, please use this template.
%
% To familiarize yourself with this template, the body contains
% some examples of its use.  Look them over.  Then you can
% run LaTeX on this file.  After you have LaTeXed this file then
% you can look over the result either by printing it out with
% dvips or using xdvi.
%

\documentclass[twoside]{article}
\setlength{\oddsidemargin}{0.25 in}
\setlength{\evensidemargin}{-0.25 in}
\setlength{\topmargin}{-0.6 in}
\setlength{\textwidth}{6.5 in}
\setlength{\textheight}{8.5 in}
\setlength{\headsep}{0.75 in}
\setlength{\parindent}{0 in}
\setlength{\parskip}{0.1 in}

%
% ADD PACKAGES here:
%

\usepackage{amsfonts,graphicx}
%\usepackage{amsmath}
\usepackage{algorithm}
%\usepackage[noend]{algpseudocode}
\usepackage{verbatim}

\usepackage[T1]{fontenc}
\usepackage[utf8]{inputenc}
\usepackage[english]{babel}
%\usepackage[margin=1in]{geometry}
\usepackage{natbib}
%\setcitestyle{nocompress}
%\bibpunct{(}{)}{;}{a}{}{,}

% \usepackage[nocompress]{cite}
%\usepackage{graphicx}
%\usepackage[cmex10]{amsmath}
%\usepackage{authblk}
% \usepackage[colorinlistoftodos,prependcaption,textsize=tiny]{todonotes}
\usepackage{macros}
\usepackage{times}


%
% The following commands set up the lecnum (lecture number)
% counter and make various numbering schemes work relative
% to the lecture number.
%
\newcounter{lecnum}
\renewcommand{\thepage}{\arabic{page}}
\renewcommand{\thesection}{\arabic{section}}
\renewcommand{\theequation}{\arabic{equation}}
\renewcommand{\thefigure}{\arabic{figure}}
\renewcommand{\thetable}{\arabic{table}}

%
% The following macro is used to generate the header.
%
\newcommand{\lecture}[4]{
   \pagestyle{myheadings}
   \thispagestyle{plain}
   \newpage
   \setcounter{lecnum}{#1}
   \setcounter{page}{1}
   \noindent
   \begin{center}
   \framebox{
      \vbox{\vspace{2mm}
    %\hbox to 6.28in { {\bf Report on my work with Adobe
		%\hfill Spring 2018} }
       \vspace{4mm}
       \hbox to 6.28in { {\Large \hfill Report on my work with Adobe \hfill} }
       \vspace{2mm}
       \hbox to 6.28in { {\it Adobe Advisor(s): Branislav Kveton, Anup Rao \hfill Intern: Subhojyoti Mukherjee} }
      \vspace{2mm}}
   }
   \end{center}
   %\markboth{Lecture #1: #2}{Lecture #1: #2}

   {\bf Disclaimer}: {\it These notes have not been subjected to the
   usual scrutiny reserved for formal publications.  They may be distributed
   outside this communication only with the permission of the Advisor(s).}
   \vspace*{4mm}
}
%
% Convention for citations is authors' initials followed by the year.
% For example, to cite a paper by Leighton and Maggs you would type
% \cite{LM89}, and to cite a paper by Strassen you would type \cite{S69}.
% (To avoid bibliography problems, for now we redefine the \cite command.)
% Also commands that create a suitable format for the reference list.
\renewcommand{\cite}[1]{[#1]}
\def\beginrefs{\begin{list}%
        {[\arabic{equation}]}{\usecounter{equation}
         \setlength{\leftmargin}{2.0truecm}\setlength{\labelsep}{0.4truecm}%
         \setlength{\labelwidth}{1.6truecm}}}
\def\endrefs{\end{list}}
\def\bibentry#1{\item[\hbox{[#1]}]}

%Use this command for a figure; it puts a figure in wherever you want it.
%usage: \fig{NUMBER}{SPACE-IN-INCHES}{CAPTION}
\newcommand{\fig}[3]{
			\vspace{#2}
			\begin{center}
			Figure #1:~#3
			\end{center}
	}
% Use these for theorems, lemmas, proofs, etc.
%\newtheorem{theorem}{Theorem}
%\newtheorem{lemma}[theorem]{Lemma}
%\newtheorem{proposition}[theorem]{Proposition}
%\newtheorem{claim}[theorem]{Claim}
%\newtheorem{corollary}[theorem]{Corollary}
%\newtheorem{definition}[theorem]{Definition}
%\newenvironment{proof}{{\bf Proof:}}{\hfill\rule{2mm}{2mm}}

% **** IF YOU WANT TO DEFINE ADDITIONAL MACROS FOR YOURSELF, PUT THEM HERE:

%\newcommand\E{\mathbb{E}}


\begin{document}
%FILL IN THE RIGHT INFO.
%\lecture{**LECTURE-NUMBER**}{**DATE**}{**LECTURER**}{**SCRIBE**}
\lecture{1}{A - Title}{Lecturer Name}{scribe-name}
%\footnotetext{These notes are partially based on those of Nigel Mansell.}

% **** YOUR NOTES GO HERE:

% Some general latex examples and examples making use of the
% macros follow.  
%**** IN GENERAL, BE BRIEF. LONG SCRIBE NOTES, NO MATTER HOW WELL WRITTEN,
%**** ARE NEVER READ BY ANYBODY.
%This lecture's notes illustrate some uses of
%various \LaTeX\ macros.  
%Take a look at this and imitate.

\section{Summary Of Discussion}

	We have proposed and deliberated on a generalized latent bandit model.
	
\section{Related Works}

	In \citet{maillard2014latent} the authors propose the Latent Bandit model where there are two sets: 1) set of arms denoted by $\A$ and 2) set of types denoted by $\B$ which contains the latent information regarding the arms. The latent information for the arms are modeled such that the set $\B$ is assumed to be partitioned into $|C|$ clusters, indexed by $\B_1, \B_2, \ldots, \B_C \in \C$ such that the distribution $v_{a,b}, a\in\A, b\in\B_c$ across each cluster is same.  Note, that the identity of the cluster is unknown to the learner. At every timestep $t$, nature selects a type $b_t\in\B_c$ and then the learner selects an arm $a_t\in\A$ and observes a reward $X_{a,t}$ from the distribution $v_{a,b}$.
	
	Another way to look at this problem is to imagine a matrix of dimension $|A|\times |B|$ where again the rows in $\B$ can be partitioned into $|C|$ clusters, such that the distribution across each of this clusters are same. Now, at every timestep $t$ one of this row is revealed to the learner and it chooses one column such that the $v_{a,b}$ is one of the $\lbrace v_{a,c}\rbrace_{c\in\C}$ and the reward for that arm and the user is revealed to the learner.
	
	This is actually a much simpler approach because note that the distributions across each of the clusters $\lbrace v_{a,c}\rbrace_{c\in\C}$ are identical and estimating one cluster distribution will reveal all the information of the users in each cluster.


\section{Problem Formulation: Generalized Latent Bandit Model}
	
	Extending the discussion in the previous section, we propose a Generalized Latent Bandit model where each $v_{a,b}$ can be considered as a mixture of several $\lbrace v_{a,c}\rbrace_{c\in\C}$. This is a harder problem than Latent Bandit model because now the rows in each cluster are not exactly identical but the index of the optimal arm (column) is similar in each row.
	
	To define more formally, let $R$ be matrix of  dimension $K\times L$ where $|A|=K$ is the number of rows and $L=|B|$ is the number of columns. Also, let us assume that this matrix  $R$ has a low rank structure of rank $d << \min\lbrace L,K\rbrace$. Let $U$ and $V$ denote the latent matrices which are not visible to the learner such that,
\begin{align*}
	R = UV^{T} \textbf{ \hspace*{4mm}   s.t.   \hspace*{4mm}} U\in [ 0,1 ]^{K\times d} \textbf{, } V\in  [0,1]^{L\times d} 
\end{align*}	  
	
	Furthermore, we put a constraint on $V$ such that, $\forall j\in [L]$, $ ||V(j,:)||_1 \leq 1$. 
	
	%Also, we assume that $R$ is a hott topics matrix such that there exist $d$-base row factors denoted by $U(\mathbb{I}^*,:)$ such that all rows in $U$ can be expressed as a convex combination of the rows of $U(\mathbb{I},:)$ and the zero vector. 
	
%	\textbf{Comment 1: The constraints on $V$ needs to be worked out as pointed out by Anup.}
%	
%	\textbf{Comment 2: Only row base factors are required. Why??}
	
	
\section{Approach Discussed}

The approach we discussed is in the context of noise-free environment. We considered a column (arm) elimination algorithm as like \citet{auer2010ucb}. An illustrative example is given below, 

\begin{align*}
U = \begin{bmatrix}
    1 & 0 \\
    0 & 1 \\
    0.5 & 0.5
\end{bmatrix} , V= \begin{bmatrix}
    1 & 0 \\
    0 & 1 \\
    0.5 & 0.5
\end{bmatrix} \\
\therefore R = UV^{T} = \begin{bmatrix}
    1 & 0 & 0.5\\
    0 & 1 & 0.5\\
    0.5 & 0.5 & 0.5
\end{bmatrix} 
\end{align*}

In this noise-free setup, we first choose the row having a column which dominates other entries in the column (here col{1}). If there is a clash  in the rows, we choose arbitrarily among them. We then eliminate the dominating column in the row. Then we follow this process iteratively for other surviving columns till we have eliminated $d$ columns. 

So, in the above setup, we will be having two passes giving us col{1,2}.

%\textbf{Comment 3: In the noisy scenario we need to find a similar algorithm}.


\section{Assumptions}

\begin{assumption}
\label{assm:1}
We assume that there exists $d$-column base factors, denoted by $V(J^*,:)$, such that all row of $V$ can be written as a convex combination of $V(J^*,:)$ and the zero vector and $J^* = [d]$. We denote the column factors by $V^*$. Therefore, for any $v\in V$, it can be represented by
\begin{align*}
v = aV^* , 
\end{align*}
where $a\in [0,1]^{d}$ and $ ||a||_1 \leq 1$.
\end{assumption}

\section{Main Result and Proofs}

\begin{lemma}
For any arbitrary row $i\in[K]$, 
\begin{align*}
\argmax_{j\in[L]} U(i,:){V(j,:)}^T \leq \argmax_{j\in[d]} U(i,:){V(j,:)}^T . 
\end{align*}
\end{lemma}

\begin{proof}
Considering any arbitrary row $i\in [K]$, we can show that,
\begin{align*}
\argmax_{j\in[L]} U(i,:){V(j,:)}^T & \overset{(a)}{\leq} \argmax_{j\in[L]} U(i,:)(aV*)^T \\
& \leq \argmax_{j\in[d]}U(i,:){V(j,:)}^T ,
\end{align*}
where $(a)$ is from Assumption \ref{assm:1}.
\end{proof}



%\clearpage
%\newpage
%\bibliographystyle{plainnat}
\bibliographystyle{apalike}
\bibliography{biblio}

\end{document}





