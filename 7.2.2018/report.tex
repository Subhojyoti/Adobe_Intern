%
% This is the LaTeX template file for lecture notes for CS267,
% Applications of Parallel Computing.  When preparing 
% LaTeX notes for this class, please use this template.
%
% To familiarize yourself with this template, the body contains
% some examples of its use.  Look them over.  Then you can
% run LaTeX on this file.  After you have LaTeXed this file then
% you can look over the result either by printing it out with
% dvips or using xdvi.
%

\documentclass[twoside]{article}
\setlength{\oddsidemargin}{0.25 in}
\setlength{\evensidemargin}{-0.25 in}
\setlength{\topmargin}{-0.6 in}
\setlength{\textwidth}{6.5 in}
\setlength{\textheight}{8.5 in}
\setlength{\headsep}{0.75 in}
\setlength{\parindent}{0 in}
\setlength{\parskip}{0.1 in}

%
% ADD PACKAGES here:
%

\usepackage{amsfonts,graphicx}
%\usepackage{amsmath}
\usepackage{algorithm}
%\usepackage[noend]{algpseudocode}
\usepackage{verbatim}

\usepackage[T1]{fontenc}
\usepackage[utf8]{inputenc}
\usepackage[english]{babel}
%\usepackage[margin=1in]{geometry}
\usepackage{natbib}
%\setcitestyle{nocompress}
%\bibpunct{(}{)}{;}{a}{}{,}

% \usepackage[nocompress]{cite}
%\usepackage{graphicx}
%\usepackage[cmex10]{amsmath}
%\usepackage{authblk}
% \usepackage[colorinlistoftodos,prependcaption,textsize=tiny]{todonotes}
\usepackage{macros}


%
% The following commands set up the lecnum (lecture number)
% counter and make various numbering schemes work relative
% to the lecture number.
%
\newcounter{lecnum}
\renewcommand{\thepage}{\arabic{page}}
\renewcommand{\thesection}{\arabic{section}}
\renewcommand{\theequation}{\arabic{equation}}
\renewcommand{\thefigure}{\arabic{figure}}
\renewcommand{\thetable}{\arabic{table}}

%
% The following macro is used to generate the header.
%
\newcommand{\lecture}[4]{
   \pagestyle{myheadings}
   \thispagestyle{plain}
   \newpage
   \setcounter{lecnum}{#1}
   \setcounter{page}{1}
   \noindent
   \begin{center}
   \framebox{
      \vbox{\vspace{2mm}
    \hbox to 6.28in { {\bf Report on my work with Adobe
		\hfill Spring 2018} }
       \vspace{4mm}
       \hbox to 6.28in { {\Large \hfill Report on my work with Adobe  \hfill} }
       \vspace{2mm}
       \hbox to 6.28in { {\it Adobe Advisor: Branislav Kveton \hfill Intern: Subhojyoti Mukherjee} }
      \vspace{2mm}}
   }
   \end{center}
   %\markboth{Lecture #1: #2}{Lecture #1: #2}

   {\bf Disclaimer}: {\it These notes have not been subjected to the
   usual scrutiny reserved for formal publications.  They may be distributed
   outside this communication only with the permission of the Advisor.}
   \vspace*{4mm}
}
%
% Convention for citations is authors' initials followed by the year.
% For example, to cite a paper by Leighton and Maggs you would type
% \cite{LM89}, and to cite a paper by Strassen you would type \cite{S69}.
% (To avoid bibliography problems, for now we redefine the \cite command.)
% Also commands that create a suitable format for the reference list.
\renewcommand{\cite}[1]{[#1]}
\def\beginrefs{\begin{list}%
        {[\arabic{equation}]}{\usecounter{equation}
         \setlength{\leftmargin}{2.0truecm}\setlength{\labelsep}{0.4truecm}%
         \setlength{\labelwidth}{1.6truecm}}}
\def\endrefs{\end{list}}
\def\bibentry#1{\item[\hbox{[#1]}]}

%Use this command for a figure; it puts a figure in wherever you want it.
%usage: \fig{NUMBER}{SPACE-IN-INCHES}{CAPTION}
\newcommand{\fig}[3]{
			\vspace{#2}
			\begin{center}
			Figure #1:~#3
			\end{center}
	}
% Use these for theorems, lemmas, proofs, etc.
%\newtheorem{theorem}{Theorem}
%\newtheorem{lemma}[theorem]{Lemma}
%\newtheorem{proposition}[theorem]{Proposition}
%\newtheorem{claim}[theorem]{Claim}
%\newtheorem{corollary}[theorem]{Corollary}
%\newtheorem{definition}[theorem]{Definition}
%\newenvironment{proof}{{\bf Proof:}}{\hfill\rule{2mm}{2mm}}

% **** IF YOU WANT TO DEFINE ADDITIONAL MACROS FOR YOURSELF, PUT THEM HERE:

%\newcommand\E{\mathbb{E}}


\begin{document}
%FILL IN THE RIGHT INFO.
%\lecture{**LECTURE-NUMBER**}{**DATE**}{**LECTURER**}{**SCRIBE**}
\lecture{1}{A - Title}{Lecturer Name}{scribe-name}
%\footnotetext{These notes are partially based on those of Nigel Mansell.}

% **** YOUR NOTES GO HERE:

% Some general latex examples and examples making use of the
% macros follow.  
%**** IN GENERAL, BE BRIEF. LONG SCRIBE NOTES, NO MATTER HOW WELL WRITTEN,
%**** ARE NEVER READ BY ANYBODY.
%This lecture's notes illustrate some uses of
%various \LaTeX\ macros.  
%Take a look at this and imitate.

\section{Summary Of Discussion}

	I have been doing literature survey on the area chosen: \textbf{Learning latent variable models through online learning}. 
	
\section{Problem Definition: Latent variable model}
	
	
	
	
	
\section{Notations}
$T$ denotes the time horizon. $\mathcal{A}$ denotes the set of arms with individual arm is denoted by $i$ such that $i=1,\ldots, K$. The term $S_i$ and $F_i$ denotes the success and the failure respectively for the arm $i$ . The distribution associated with individual arm $i$ is denoted by $D_i$ whereas the reward drawn from that distribution for the $t$-th time instant is denoted by $x_{i,t}$. $r_i$ denotes the expected mean of the reward distribution $D_i$ and $\hat{r}_{i}(t)$ denotes the estimated mean for the arm $i$. All rewards are bounded in $[0,1]$. $n_i$ denotes the number of times arm $i$ has been pulled.

	We define each expert or forecaster as $f_{j}\in \mathcal{M}_t$, where $\mathcal{M}_t$ is the set of all forecasters at time $t$. $\mathcal{M^+}_t$ is the set of new forecasters introduced at time $t$. Also, we define $\hat{L}_{f_i,t}=\sum_{s=1}^{t}\ell_{f_i ,s}$ as the true cumulative loss suffered by the expert $f_i$ till $t$-th timestep and $\hat{L}_{f_i,t}$ as the estimated cumulative loss suffered by an expert $f_i$ till $t$-th timestep such that $\hat{L}_{f_i,t}=\sum_{s=1}^{t}\hat{\ell}_{f_i ,s}$. Similarly, we define $L_{i}$ and $\hat{L}_{i}$ as the true loss and the estimated loss suffered by arm $i$. The weight of an expert $f_i$ at time $t$ is defined as $w_{f_i,t}$ and $\eta$ is defined as a parameter for exploration. Also let $i_{f_j\in \mathcal{M},t}$ be the action suggested by expert $f_j$ at time $t$.


%\section{Algorithms}
%\subsection{Noise-Free Setting}

In the noise-free setting, in addition to Assumption \ref{assm:1} and Assumption \ref{assm:1} we assume two further assumptions.

\begin{assumption}
\label{assm:round-robin}
We assume that nature is revealing $i$ in $\bar{R}(i,:), \forall i\in [K]$  in a Round-Robin fashion.
\end{assumption}

%\begin{assumption}
%\label{assm:d-indep}
%We assume that any $d$ sets of rows or columns are independent in $\bar{R}$.
%\end{assumption}

\algblock{ArmElim}{EndArmElim}
\algnewcommand\algorithmicArmElim{\textbf{\em Arm Elimination}}
 \algnewcommand\algorithmicendArmElim{}
\algrenewtext{ArmElim}[1]{\algorithmicArmElim\ #1}
\algrenewtext{EndArmElim}{\algorithmicendArmElim}

\algblock{ResParam}{EndResParam}
\algnewcommand\algorithmicResParam{\textbf{\em Reset Parameters}}
 \algnewcommand\algorithmicendResParam{}
\algrenewtext{ResParam}[1]{\algorithmicResParam\ #1}
\algrenewtext{EndResParam}{\algorithmicendResParam}

\algblock{ColRec}{EndColRec}
\algnewcommand\algorithmicColRec{\textbf{\em Column Reconstruction}}
 \algnewcommand\algorithmicendColRec{}
\algrenewtext{ColRec}[1]{\algorithmicColRec\ #1}
\algrenewtext{EndColRec}{\algorithmicendColRec}

\algblock{ColElim}{EndColElim}
\algnewcommand\algorithmicColElim{\textbf{\em Column Elimination}}
 \algnewcommand\algorithmicendColElim{}
\algrenewtext{ColElim}[1]{\algorithmicColElim\ #1}
\algrenewtext{EndColElim}{\algorithmicendColElim}

\algblock{Explore}{EndExplore}
\algnewcommand\algorithmicExplore{\textbf{\em Explore}}
 \algnewcommand\algorithmicendExplore{}
\algrenewtext{Explore}[1]{\algorithmicExplore\ #1}
\algrenewtext{EndExplore}{\algorithmicendExplore}


\algblock{Exploit}{EndExploit}
\algnewcommand\algorithmicExploit{\textbf{\em Exploit}}
 \algnewcommand\algorithmicendExploit{}
\algrenewtext{Exploit}[1]{\algorithmicExploit\ #1}
\algrenewtext{EndExploit}{\algorithmicendExploit}

\algnewcommand\algorithmicforeach{\textbf{for each}}
\algdef{S}[FOR]{ForEach}[1]{\algorithmicforeach\ #1\ \algorithmicdo}


\begin{algorithm}[!th]
\caption{Noise-Free GLB}
\label{alg:NFGLB}
\begin{algorithmic}[1]
\State {\bf Input:} Time horizon $T$, $Rank(\bar{R}) = d$.
\State {\bf Explore Parameter:} $\gamma = \lceil\sqrt{K} \rceil$.
\State {\bf Initialization:} $\forall I\in [K], J\in [L], \hat{R}(I,J) \leftarrow 0$, $m=0$, $\B_0 \leftarrow \A$, $\Z_0 \leftarrow \emptyset$, $c=0$ , $n_0 = 1 $ and $N_0 = \gamma |\B_0|n_0$.
\State Create set $\C$ of equivalence classes such that 
\ForEach{$b\in \left[0,\frac{K}{\gamma}-1\right]$}
\State  $e_{b+1}\in C \leftarrow \left\lbrace i\in \left[\frac{bK}{\gamma}+1,\frac{(b+1)K}{\gamma}\right]\right\rbrace$.
\EndFor
\For{$t=1,..,T$}	
\State Nature reveals $i_t$ such that $i_t \leftarrow (t \mod K) + 1$ (Round-Robin).
\If{$|\B_m| > d$} \textbf{ (Explore) }
\If{$t \geq N_m$}
\If{$r_c \leq \gamma $}
\State Choose $c$, observe $\bar{R}(r_c,c)$ and $\hat{R}(r_c,c)\leftarrow \bar{R}(r_c,c)$.
\State $r_c \leftarrow r_c + 1$.
\Else
\State $\Z_m \leftarrow \Z_m \cup \lbrace c\rbrace$
\State Choose $c\in\B_m\setminus \Z_m$ uniform randomly and $r_c \leftarrow 1$.
\EndIf
\Else
\ColElim
\State \ForEach{$e_b\in\C$}
\State \If{$\exists j\in \B_m: \forall i\in e_b, \forall j'\in \B_m\setminus \lbrace j\rbrace,  \hat{R}(i,j) < \max_{i\in e_b}\lbrace \hat{R}(i,j') \rbrace$}
\State $\B_m \leftarrow \B_m \setminus \lbrace j \rbrace$. 
\EndIf
\EndFor
\EndColElim
\ResParam
\State $N_{m+1} \leftarrow t + \gamma |\B_m|n_0$ and $m \leftarrow m + 1$.
\EndResParam
\EndIf
\Else \textbf{  (Exploit) }
\State  Select column $j_t^*$, observe $\bar{R}(i_t,j_t^*)$ where $j_t^* \leftarrow \argmax_{j\in[\B_m]} {\hat{R}(i_t,j)}$ and $\hat{R}(i_t,,j_t^*)\leftarrow \bar{R}(i_t,,j_t^*)$.
\EndIf
\EndFor
\end{algorithmic}
%\vspace*{-0.42em}
\end{algorithm}



\begin{algorithm}[!th]
\caption{Noisy GLB-UCB}
\label{alg:NGLB}
\begin{algorithmic}[1]
\State {\bf Input:} Time horizon $T$, $Rank(\bar{R}) = d$.
\State {\bf Definition:} $U_m(\epsilon_m, n_m) = \sqrt{\dfrac{2\log(T\epsilon_m^2)}{n_m} }$
\State {\bf Explore Parameter:} $\gamma = \lceil\sqrt{K} \rceil$.
\State {\bf Initialization:} $\forall I\in [K], J\in [L], \hat{R}(I,J) \leftarrow 0$, $m=0$, $\B_0 \leftarrow \A$, $\Z_0 \leftarrow \emptyset$, $c=0$ , $n_0 = 1 $ and $N_0 = \gamma |\B_0|n_0$.
\State Create set $\C$ of equivalence classes such that 
\ForEach{$b\in \left[0,\frac{K}{\gamma}-1\right]$}
\State  $e_{b+1}\in C \leftarrow \left\lbrace i\in \left[\frac{bK}{\gamma}+1,\frac{(b+1)K}{\gamma}\right]\right\rbrace$.
\EndFor
\For{$t=1,..,T$}	
\State Nature reveals $i_t$ such that $i_t \leftarrow (t \mod K) + 1$ (Round-Robin).
\If{$|\B_m| > d$} \textbf{ (Explore) }
\If{$t \geq N_m$}
\If{$r_c \leq \gamma $}
\State Choose $c$, observe $\bar{R}(r_c,c)$ and $\hat{R}(r_c,c)\leftarrow \bar{R}(r_c,c)$.
\State $r_c \leftarrow r_c + 1$.
\Else
\State $\Z_m \leftarrow \Z_m \cup \lbrace c\rbrace$
\State Choose $c\in\B_m\setminus \Z_m$ uniform randomly and $r_c \leftarrow 1$.
\EndIf
\Else
\ColElim
\State \ForEach{$e_b\in\C$}
\State \If{$\exists j\in \B_m: \forall i\in e_b, \forall j'\in \B_m\setminus \lbrace j\rbrace, \hat{R}(i,j)  + U_m(\epsilon_m, n_m) < \max_{i\in e_b}\lbrace \hat{R}(i,j') - U_m(\epsilon_m, n_m)\rbrace$}
\State $\B_m \leftarrow \B_m \setminus \lbrace j \rbrace$. 
\EndIf
\EndFor
\EndColElim
\ResParam
\State $\epsilon_{m+1} \leftarrow \frac{\epsilon_m}{2}$ and $n_{m+1} \leftarrow \dfrac{2\log(T\epsilon_{m+1}^2)}{\epsilon_{m+1}^2} $
\State $N_{m+1} \leftarrow t + \gamma |\B_m|n_{m+1}$ and $m \leftarrow m + 1$.
\EndResParam
\EndIf
\Else \textbf{  (Exploit) }
\State  Select column $j_t^*$, observe $\bar{R}(i_t,j_t^*)$ where $j_t^* \leftarrow \argmax_{j\in[\B_m]} {\hat{R}(i_t,j)}$ and $\hat{R}(i_t,,j_t^*)\leftarrow \bar{R}(i_t,,j_t^*)$.
\EndIf
\EndFor
\end{algorithmic}
%\vspace*{-0.42em}
\end{algorithm}



%
%
%\section{Experiments}
%\begin{figure}[!th]
    \centering
    \begin{tabular}{cc}
    \setlength{\tabcolsep}{0.1pt}
    \subfigure[0.25\textwidth][Expt-$1$: $3$ Users, $3$ Bernoulli-distributed arms and Round-Robin]
    %with $r_{i_{{i}\neq {*}}}=0.07$ and $r^{*}=0.1$
    {
    		\pgfplotsset{
		tick label style={font=\Large},
		label style={font=\Large},
		legend style={font=\Large},
		ylabel style={yshift=5pt},
		%legend style={legendshift=32pt},
		}
        \begin{tikzpicture}[scale=0.8]
      	\begin{axis}[
		xlabel={timestep},
		ylabel={Cumulative Regret},
		grid=major,
        %clip mode=individual,grid,grid style={gray!30},
        clip=true,
        %clip mode=individual,grid,grid style={gray!30},
  		legend style={at={(0.5,1.4)},anchor=north, legend columns=3} ]
      	% UCB
		\addplot table{results/NewExpt/Expt2/comp_subsampled_CustomUCB0RR1.txt};
		\addplot table{results/NewExpt/Expt2/comp_subsampled_CustomEXP30RR1.txt};
      	\legend{UCB1-RR,EXP3-RR}      	
      	\end{axis}
      	\end{tikzpicture}
  		\label{fig:1}
    }
    &
    \subfigure[0.25\textwidth][Expt-$2$: $3$ Users, $3$ Bernoulli-distributed arms and Uniform Sampling ]
    %with $r_{i_{{i}\neq {*}}}=0.07$ and $r^{*}=0.1$
    {
    		\pgfplotsset{
		tick label style={font=\Large},
		label style={font=\Large},
		legend style={font=\Large},
		ylabel style={yshift=5pt},
		%legend style={legendshift=32pt},
		}
        \begin{tikzpicture}[scale=0.8]
      	\begin{axis}[
		xlabel={timestep},
		ylabel={Cumulative Regret},
		grid=major,
        %clip mode=individual,grid,grid style={gray!30},
        clip=true,
        %clip mode=individual,grid,grid style={gray!30},
  		legend style={at={(0.5,1.4)},anchor=north, legend columns=3} ]
      	% UCB
		\addplot table{results/NewExpt/Expt2/comp_subsampled_CustomUCB0US1.txt};
		\addplot table{results/NewExpt/Expt2/comp_subsampled_CustomEXP30US1.txt};
      	\legend{UCB1-US,EXP3-US}      	
      	\end{axis}
      	\end{tikzpicture}
  		\label{fig:2}
    }
    \\
    \subfigure[0.25\textwidth][Expt-$2$: $20$ Users, $3$ Bernoulli-distributed arms, Round-Robin, Noise-Free Setting ]
    %with $r_{i_{{i}\neq {*}}}=0.07$ and $r^{*}=0.1$
    {
    		\pgfplotsset{
		tick label style={font=\Large},
		label style={font=\Large},
		legend style={font=\Large},
		ylabel style={yshift=5pt},
		%legend style={legendshift=32pt},
		}
        \begin{tikzpicture}[scale=0.8]
      	\begin{axis}[
		xlabel={timestep},
		ylabel={Cumulative Regret},
		grid=major,
        %clip mode=individual,grid,grid style={gray!30},
        clip=true,
        %clip mode=individual,grid,grid style={gray!30},
  		legend style={at={(0.5,1.4)},anchor=north, legend columns=3} ]
      	% UCB
		\addplot table{results/NewExpt/Expt3/comp_subsampled_CustomUCB0RR.txt};
		\addplot table{results/NewExpt/Expt3/comp_subsampled_GLBUCB0RR.txt};
		\addplot table{results/NewExpt/Expt3/comp_subsampled_CustomEXP30RR.txt};
      	\legend{UCB1-RR,GLBUCB-RR, EXP3-RR}      	
      	\end{axis}
      	\end{tikzpicture}
  		\label{fig:3}
    }
    \end{tabular}
    \caption{A comparison of the cumulative regret incurred by the various bandit algorithms. }
    \label{fig:karmed}
    \vspace*{-1em}
\end{figure}


%\section{Main Results}
%\input{proof}

\clearpage
\newpage
%\bibliographystyle{plainnat}
\bibliographystyle{apalike}
\bibliography{biblio}

\end{document}





